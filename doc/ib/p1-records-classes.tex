\chapter{Records and Classes}
\label{RC-Chapter}

\textsc{Perspective}: Records are fixed size data aggregates where each
component (field) is named. Icon's records look superficially similar to
records in other languages but, as with other data structures, there is a
lot going on behind the scenes. Much of the complication arises because the
translator is unable to check whether a particular field name is correct
until run time. For some methods of access the field name is not known until
run time, precluding the possibility of compile time checking
altogether. Information about the order of fields and their names must
therefore be present at run time.  This information is stored in the record
descriptor block for each record, which contains the string field names in
order. For efficiency, it is also stored centrally in a global field table
which is used by the field reference operator to select the correct field in
each record.

Unicon allows new record types to be constructed at run time, which is
particularly useful in database applications. Internally, it uses a slight
extension of Icon's record type to implement classes, which add further to
the information required in the descriptor.

\section{Records}
Much of the implementation of records has already been mentioned, in
passing, in earlier sections where a record was used as one of the example
data structures -- see, for example, page \pageref{ComplexRecord}. What has
been missing, until now, is a description of the record constructor
block. Record constructors are a kind of procedure, so the record
constructor block is a special case procedure block that provides
alternative names and uses for some of the fields.

\begin{iconcode}
/*\\
 * Alternate uses for procedure block fields, applied to records.\\
 */\\
\#define nfields nparam\ \ \  /* number of fields */\\
\#define recnum  nstatic\ \ \ /* record number */\\
\#define recid   fstatic\ \ \ /* record serial number */\\
\#define recname pname \ \ \  /* record name */\\
\end{iconcode}

A central data structure called the field table (described later in section
\ref{FieldTableDescription}) allows constant time access to fields by name;
all field names used in the dot operator are compiled down to integer
subscripts within the field table, which maps their positions for all record
types known at link time.

Here is the full picture, including a record constructor, of how the example
previously given on page \pageref{ComplexRecord} looks if it is constructed
dynamically by
\begin{iconcode}
  point := constructor("complex", "r", "i")(1,3)
\end{iconcode}

\label{FullComplexRecord}
\begin{picture}(300,400)(-20,-240)
%\put(-10,-250){\graphpaper{30}{40}}
\put(0,112){\dvboxptr{record}{np}{60}{}}
\put(140,96){\blkbox{record}{32}}
\put(140,96){\rightboxlabels{title}{size of block in bytes}}
\put(140,80){\wordbox{\textit{id}}{}}
\put(140,64){\wordboxptr{50}{~~~~~record constructor}}
% Join up the record constructor block
\put(270,72){\line(1,0){10}}
\put(280,72){\line(0,-1){82}}
\put(280,-10){\vector(-1,0){160}}
\put(120,-10){\line(-1,0){140}}
\put(-20,-10){\line(0,-1){20}}
\put(-20,-30){\vector(1,0){20}}
%  
\put(140,32){\dvbox{integer}{n}{1}}
\put(140,0){\dvbox{integer}{n}{3}}
\begin{picture}(0,0)(0,150)
\put(0,16){\wordbox{1}{}}
\put(0,0){\trboxlabel{record serial number}}
\put(0,32){\blkbox{-2}{-1}}
\put(0,32){\rightboxlabels{}{record number$^{\ddagger}$}}
\put(0,64){\wordbox{2}{}}
\put(0,48){\rightboxlabels{number of fields}{record indicator$^{\dagger}$}}
\put(0,80){\blkboxptr{52}{40}{entry point of the library routine to create a record (mkrec)}}
\put(0,80){\trboxlabel{size of block}}
\put(0,112){\wordbox{proc}{}}
\begin{picture}(0,0)(0,80)
\put(0,0){\dvboxptr{1}{}{40}{"i"}}
\put(0,32){\dvboxptr{1}{}{40}{"r"}}
\put(0,64){\dvboxptr{7}{}{40}{"complex"}}
\end{picture}
\end{picture}
\end{picture}

\noindent
$^{\dagger}$The \texttt{ndynam} field (labelled as ``record indicator''
above) serves multiple purposes, depending on whether it is describing a
procedure, a builtin function, a record or a class. Its value is described
in the following table.
\begin{quote}
\begin{tabular}{cl}
  0 \ldots \textit{n} & Unicon procedure with \textit{n} parameters\\
  -1 & Builtin function\\\
  -2 & Record\\
  -3 & Class\\
\end{tabular}
\end{quote}
$^{\ddagger}$Dynamically created records have a record number of -1, which
indicates that they have no field table entry.

\section{Classes}
The ``heavy lifting'' to implement classes is done by the Unicon translator;
in the runtime system, the field reference operator provides a minor
assist. Within the runtime system, a class is, to a first approximation,
just a pair of record types. Objects are instances of a primary record type
that has some extra (hidden) fields. Thus, most of the time when the runtime
system needs to distinguish between a class object and a record, it is to
hide the extra fields. For example, in the \texttt{image} builtin function,
care is taken not to output the values of the \texttt{\_\_m} and
\texttt{\_\_s} fields. In the \texttt{size} function, they are subtracted
from the total count of fields etc. The additional code for \texttt{size} is
typical
\begin{iconcode}
operator{1} * size(x)\\
\>   abstract \{\\
\>\>      return integer\\
\>      \}\\
\>   type\_case x of \{\\
\ldots\\
\>      record: inline \{\\
\>\>         C\_integer siz;\\
\>\>	 union block *bp, *rd;\\
\>\>	 bp = BlkLoc(x);\\
\>\>	 rd = Blk(bp,Record)->recdesc;\\
\>\>         siz = Blk(BlkD(x,Record)->recdesc,Proc)->nfields;\\
\>	 {\color{blue}/*}\\
\>	 {\color{blue}~* if the record is an object, subtract 2 from the size}\\
\>	 {\color{blue}~*/}\\
\>\> {\color{blue} if (Blk(rd,Proc)->ndynam == -3)~ siz -= 2;}\\
\>\>         return C\_integer siz;\\
\>         \}\\
\ldots\\
\end{iconcode}

When referencing the fields by field number, care has to be taken to skip
over the hidden fields by adding 2 to the index.  The \texttt{?} and
\texttt{!}  operators must also take the hidden fields into account when
their operand is a class.

\subsection{Definitions made by a class declaration}
When a class (called, for example, \texttt{foo}) is declared, two record
types are created called \texttt{foo\_\_methods} and
\texttt{foo\_\_state}. \texttt{foo\_\_methods}, as the name suggests, is a
vector of procedure descriptors, one for each method declared within the
class and its superclasses.  A singleton instance of \texttt{foo\_\_methods}
is shared by all \texttt{foo} objects. \texttt{foo\_\_state} contains the
descriptors for the fields of the class, prepended with two extra fields
\texttt{\_\_s} and \texttt{\_\_m} at the start of the vector.

\texttt{\_\_m} is used by the field reference operator when selecting a
method for subsequent invocation. \texttt{\_\_s} is self reference and is
seemingly more vestigial%
%
\footnote{Unicon's predecessor Idol used \fntexttt{\_\_s} to implement
the notion that object fields were {\em private} and not accessible outside
the methods of a class. When Unicon eventually rejected this departure from
Icon's record semantics, much of the need for \fntexttt{\_\_s} vanished.
}:
its continued presence remains to be justified and it might be removed
someday.

The complete list of additional reserved identifiers introduced by a
declaration of a class called \texttt{foo} is given below
\begin{quote}
  \begin{tabular}{lp{5in}}
    \texttt{foo\_\_state} & \\
    \texttt{foo\_\_methods} & \\

    \texttt{foo\_bar} & for each method \texttt{bar} defined in the class.
    In the current implementation, these definitions allow a method to be
    called as a procedure from Icon code. When doing so, the class instance
    {\em must} be supplied as the first parameter.  E.g. given an instance
    \texttt{f} of \texttt{foo} that has a method \texttt{bar} defined with
    two parameters, it may be called by \texttt{foo\_bar(f,a,b)}, which is
    completely equivalent to the usual \texttt{f.bar(a,b)}.\\

    \texttt{fooinitialize} & The name given to the code defined by the
    \texttt{initially} section of the class.  The name is present even if
    the \texttt{initially} section is absent.\\

    \texttt{foo\_\_oprec} & The (global) name for the methods vector of
    \texttt{foo}. Used by the \texttt{oprec} standard function to search
    through the list of globals for the correct methods vector.\\

   
  \end{tabular}
\end{quote}

\section{The Field Table}
\label{FieldTableDescription}
The field table is two-dimensional array that maps field names (which are
converted into integer indices during code generation) to positions of
descriptors within all of the record types known at link time. It provides
constant-time access to record fields that is even faster than list
subscripting.

The size of the field table can be a problem when the available address
space and memory are limited; some large programs with many records or
classes%
\footnote{
The use of large class libraries, such as in programs that implement
graphical interfaces, make the space occupied by the field table dramatically
worse. Every referenced class introduces two record types, and while a
program might not itself use that many fields, the methods of all referenced
classes are also fields -- they have to be because the compiler cannot know
whether \fntexttt{rec.f(1)} is actually using a field called \fntexttt{f} to
store a procedure or whether it is invoking the \fntexttt{f} method of
\fntexttt{rec} -- and there are lots of programs that contain hundreds, if
not thousands of methods.}
%
have even failed to run on small-memory platforms due to space
requirements. The problem is now a potential performance issue rather than a
matter of crashing with an out of memory error.

The field table, which is placed in the icode by the back end code generator
and linker \texttt{icont}, is a two dimensional ($N{\times}M$) array of
numbers where $M$ is the total number of record types -- including records
defined by class definitions -- and $N$ is the total number of field
names. The array contains the field offset, an index into an array of slots
in the record, for each field name in each record type that is in use%
\footnote{
Record (or class) declarations that are not used anywhere are optimized
by the linker and omitted from the field table, but this optimization may
be conservative and imperfect.
}
in the program. If a particular record type does not have that field, -1 is
stored at that position in the table.
For example
\begin{iconcode}
record a (m, n, o, p)\\
record b (o, p, q, r)\\
record c (s, t, u, m)\\
\end{iconcode}
has a resulting field table of
{\small
\begin{quote}
\begin{tabular}{|c|c|c|c|}
    \hline 
       & a  & b  & c  \\ \hline
    m  & 0  & -1 &  3 \\ \hline
    n  & 1  & -1 & -1 \\ \hline
    o  & 2  &  0 & -1 \\ \hline
    p  & 3  &  1 & -1 \\ \hline
    q  & -1 &  2 & -1 \\ \hline
    r  & 1  &  3 & -1 \\ \hline
    s  & -1 & -1 &  0 \\ \hline
    t  & -1 & -1 &  1 \\ \hline
    u  & -1 & -1 & -2 \\ \hline
 \end{tabular}
\end{quote}
}
The field table also includes the fields from the record types that are
generated as a result of class declarations, so both the names of the fields
in a class and the names of the class methods contribute to $N$ (and each
class declaration adds 2 to $M$). It is instructive to compare the field
table above with the corresponding example where the records have been
converted to classes (and one method added)
\begin{iconcode}
class a (m, n, o, p); method x(); end; end\\
class b (o, p, q, r); end\\
class c (s, t, u, m); end\\
\end{iconcode}
The field table that results is
{\small
\begin{quote}
 \begin{tabular}{|c|c|c|c|c|c|c|}
   \hline
& a\_\_state & a\_\_methods & b\_\_state & b\_\_methods & c\_\_state & c\_\_methods\\
\hline
\_\_s &  0 & -1 &  0 & -1 &  0 & -1 \\ \hline
\_\_m &  1 & -1 &  1 & -1 &  1 & -1 \\ \hline
    m &  2 & -1 & -1 & -1 &  5 & -1 \\ \hline
    n &  3 & -1 & -1 & -1 & -1 & -1 \\ \hline
    o &  4 & -1 &  2 & -1 & -1 & -1 \\ \hline
    p &  5 & -1 &  3 & -1 & -1 & -1 \\ \hline
    x & -1 &  0 & -1 & -1 & -1 & -1 \\ \hline
    q & -1 & -1 &  4 & -1 & -1 & -1 \\ \hline
    r & -1 & -1 &  5 & -1 & -1 & -1 \\ \hline
    s & -1 & -1 & -1 & -1 &  2 & -1 \\ \hline
    t & -1 & -1 & -1 & -1 &  3 & -1 \\ \hline
    u & -1 & -1 & -1 & -1 &  4 & -1 \\ \hline
\end{tabular}   
\end{quote}
}
The number of columns is increased, because of the extra records, and method
names are intermixed with field names.

The information needed for the field table is accumulated by the linker as
it processes the ucode files that make up the program. When it encounters a
record declaration the name of each field is stored (together with its
offset within the record type). Every time the linker encounters a
\texttt{field} instruction, it replaces the name of the field with the
integer position of that name in the list of field names. Some field
references may not be contained in any declared record type -- because it
might be a dynamically created record, or it could just be plain wrong -- in
this case the linker adds the field name to the global list of field names
but does not add the field name to the field table and replaces the name of
the field with the position of the name, relative to {\em the end} of the
name list (so the number is negative). The negative number is a signal to
the field reference operator that the field table is not in use for this
access.

Although the field table is fast, it is sparse.  Programs may have many
record types, most of which will not share field names, so the table will
contain a high proportion of -1 entries corresponding to all the types where
a particular field name is not defined. Unicon has further optimizations
that can reduce the size of the table. It does this first by reducing the
size of each entry: if no record in the entire program has more than 254
fields then each element of the field table can be a byte, rather than a 16
bit (or 32 bit) integer -- although one bad actor with 255 fields will spoil
it for everyone. Secondly, it reduces the number of -1 entries in the table
by suppressing the leading and trailing -1 entries in each row and using
some clever indexing and packing.

UTR23 has the details of the compression mechanisms, which are enabled by
building the Unicon system with \texttt{FieldTableCompression} defined. Note
that the format of the icode file is different when the compression is
enabled and has a different icode version -- programs built with compression
enabled are not executable on systems that do not have compression enabled,
and vice-versa.

On modern platforms with large amounts of available memory, the pressing
need to ``squeeze'' the program into the space available has receded; but as
Wirth%
\footnote{
Niklaus Wirth, discussing the programming language (and operating system) \em{Oberon}.
} once wisely remarked
\begin{quote}
We do not consider it good engineering practice to consume a resource lavishly
just because it happens to be cheap.
\end{quote}
economical designs are still worth pursuing.

The \texttt{FieldTableCompression} optimizations described in UTR23 are not
enabled by default because they incur a small run-time penalty from the
extra indexing that is needed to implement the compression. The option is in
the source code in case Unicon is needed on some platform where space once
again becomes an overriding concern.

\section{The field reference operator}
The field reference operator, as with many other operators in the runtime
system, conceals some complexity behind its apparently simple
appearance. Exactly the same code is emitted to access the fields of a
record (both declared records and dynamically created records), for access
to the fields of a class and also for access to the class methods: it is up
to the field reference operator to sort out what is needed.

The operator is called with two parameters on the stack: a descriptor for
the record (or class) that is being referenced and an integer that describes
which field is required.
In the most straightforward case -- the linker has replaced the field name
by a positive number $n$, so the field name is described by row $n$ of the
field table -- the operator recovers the record sequence number $m$ from the
record descriptor which gives the column number in the field table.  The
field offset may then be retrieved directly from the field table.
\begin{iconcode}
  fnum = ftabp[IntVal(Arg2) * *records + Blk(rp->recdesc,Proc)->recnum - 1];
\end{iconcode}
\texttt{Arg2} is the second parameter ($n$); \texttt{*records} contains the
number of record types in the field table; the \texttt{recnum} field from the
descriptor determines $m$.

The record descriptor that is supplied for classes is the
\texttt{class\_\_state} record, so field access for classes works exactly the
same as for records. But when the access is for a method, the field offset
returned by the calculation above will be -1, because the method name is not
(usually%
\footnote{Unicon does not prohibit the name of a field in a class being the
same name as a method. But having such a field means that the method can
never be accessed via the field reference operator -- the field will be
returned instead of the method -- the method may still be accessed using its
global name via the \fntexttt{foo\_methodname( \ldots)} syntax.
}%
) in the \texttt{class\_\_state} record. In this case the operator checks if
the record is actually a class and, if so, tries again, but with the
\texttt{\_\_m} field of the class as the record descriptor. Only if this
also returns -1 will an error be raised.

The final case is when the linker has not been able to determine a field
name entry in the field table from its collection of record declarations. In
this case, it places a negative number in the second parameter and this is
used at run time to pick up the name from the global list of field names. If
the record has been dynamically constructed, the field reference operator
then makes a sequential search through the list of field names (see the
diagram on page \pageref{FullComplexRecord}) using string comparison to
determine whether or not the field name belongs to the record.

\bigskip\bigskip
\textsc{Retrospective}:
Access to the fields of a class is the same as for a record. The field table
provides excellent speed access for record types and field names known at
link time. Dynamic record types resort to sequential search and string
comparison.  Some extra code is needed to conceal the ``hidden'' fields of a
class, but the total increase is small.

\bigskip

\noindent\textbf{EXERCISES}

\noindent\textbf{\ref*{RC-Chapter}.1}
Discuss the pros and cons of replacing the two record definitions for a
class by a single definition that contains both the field names and the
method names.
%% % + pro   - con   ~ neutral
%% +  The field table will be smaller by as much as 50 percent.
%%
%% +  Method access will be slightly quicker and code implementing the field
%%    selection operator will become slightly simpler.
%%
%% ~  The compiler will have to detect, and handle, the situation of the same
%%    name in use for a field and a method (perhaps issue a warning).
%%
%% ~  The icode version will need to change (because the code in imisc.r
%%    that implements the field reference operator will have to change).
%%
%% -  The memory cost per instance might increase by many times (for classes
%%    with many methods and few fields) or only slightly (for classes with
%%    many fields and few methods).
%%
%% -  Existing code that handles the insertion of the self parameter for
%%    method invocation might be more difficult to implement if there is
%%    no ready distinction between a field and a method.
%%
%% -  Existing code (in package lang) that expects there to be two records
%%    must be changed, or it must become sensitive to the icode version.

\noindent\textbf{\ref*{RC-Chapter}.2}
Could Unicon provide dynamically created classes? What would be the advantages?
%% Unicon allows the dynamic introduction of record types, but
%% new code (for method bodies) has to be compiled.
%% It is possible to invoke the translator and then use the load()
%% function to introduce dynamic classes, but is not straightforward.
