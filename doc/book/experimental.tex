\chapter{Experimental Features}

The designers of Unicon have taken a very conservative approach when adding to
the language and when changing existing features. With the small number of
exceptions that have been previously noted on page \pageref{Unicon-Icon}, an
Icon program that runs on the final version of Icon (version 9.5, first released
in 1996) will run on the current Unicon system {\em and give the same results\/}
a quarter of a century later. The conservative approach is continued when
dealing with additions to Unicon; breaking existing Unicon programs by making an
incompatible change to the language is, in most circumstances, considered to be
a very bad thing to do.

Most of the development of Unicon starting from its progenitor has already been
discussed but there are some more experimental features that are waiting in the
wings. Some of them may never see the light of day in their present form -- or,
perhaps, in any form -- so the most cautious approach is not to rely on any of
them until they make their way from this appendix into the definition of the
language in Appendix A.

The experimental features are not usually enabled by default in a
release build of Unicon -- they can only be accessed by making the appropriate
pre-processor definitions (or, in some cases, by specifying additional arguments
to \texttt{configure}) and rebuilding the system from the source code. Some
features that are now part of the language -- for example, the array extension
to lists that makes them faster in many cases -- are still guarded by
pre-processor definitions, showing their pedigree as experimental additions, but
are now enabled by default. The plugin mechanism and the installed plugins are
an exception to the general rule: they {\em are\/} enabled by default but should
still be considered experimental and subject to change.

A Unicon Technical Report (UTR) is the preferred vehicle for introducing a
change to Unicon.  The UTR, and associated code, may undergo several rounds of
revision before being considered ready for adoption. When it is (ready), the UTR
material will usually be incorporated somewhere into the book. UTRs may be found
at \url{http://unicon.org/reports.html}. Note that {\em all\/} of the reports
are there, including those that have served their purpose and are no longer
under active development.

%% Candidates for inclusion
%%
%% UTF-8

\section{User defined operators}
This feature extends the syntax of classes to allow the built-in operator
symbols to be redefined when their operands are objects. It may be enabled by
using the \texttt{-{}-enable-ovld} option to \texttt{configure} before
rebuilding the Unicon system.

\section{Extensions to \texttt{\&random}}
This feature allows the programmer to choose from a portfolio of different
random number generators (in addition to the one provided by Icon). It is also
possible to implement other generators and use them without rebuilding Unicon.
More than one generator may be in use at the same time.
It may be enabled by defining the C preprocessor symbol \texttt{RngLibrary}
before rebuilding the Unicon system.

\section{Plugins}
\index{Plugins}
A Unicon plugin is a dynamically loaded library of routines that are
encapsulated by a class, which provides access to the external routines (often
written in another language) plus a simple facility to enumerate the routines
that are available. The external routines may be part of the source code of the
plugin or in a separately compiled library that is acquired and installed from
elsewhere. At the minimum, a plugin provides a translation layer that converts
the arguments to the routines from their Unicon representation into something
compatible with the calling conventions of the external implementation language;
but usually the plugin also ``adds value'' by providing a more Unicon-like way
of accessing the underlying routines.

\subsection{Bitman}
\index{Bitman plugin}
The \texttt{Bitman} plugin provides low level bit manipulation routines that are
the equivalent of the built-in functions (\texttt{iand}, \texttt{ior},
\texttt{ishift} etc.). The salient differences -- and the reason for their
existence -- between the \texttt{Bitman} methods and the built-in functions is
that the \texttt{Bitman} methods never produce a large integer, are confined to
the natural word length of the machine (so the results are not portable between
32-bit and 64-bit implementations) and, except for bit shifting, do not make a
special case of the sign bit.  \texttt{Bitman} also provides some bit level
enquiry and extraction methods that use a similar addressing convention to
string indexing (indexing bits instead of characters).

The methods provided by the plugin are
\begin{center}
\begin{xtabular}{|m{0.5in}|m{2.5in}|m{0.5in}|}
    \hline
    \texttt{band}   & bitwise AND              & (\texttt{iand})\\
    \texttt{bcom}   & bitwise one's complement & (\texttt{icom})\\
    \texttt{bit}    & (single) bit extraction  &\\
    \texttt{bits}   & enquiry and multi-bit extraction &\\
    \texttt{bor}    & bitwise inclusive OR     & (\texttt{ior})\\
    \texttt{brot}   & bit rotation             &\\
    \texttt{bshift} & bitwise shift            & (\texttt{ishift})\\
    \texttt{bxor}   & bitwise exclusive OR     & (\texttt{ixor})\\
    \texttt{ushift} & unsigned bitwise shift   &\\
    \texttt{test}   & confidence testing       &\\
    \hline
\end{xtabular}
\end{center}
The actual bit manipulation code (written in C) is part of the source code for
the plugin: no external libraries are required. The \texttt{Bitman} class has no
attributes, so the constructor function has no parameters.
\bigskip\hrule\vspace{0.1cm}
\index{band}
\noindent
{\bf band(i, i) : integer } \hfill {\bf bitwise and}

\noindent
\index{bitwise and}\texttt{band(i1, i2)} produces the bitwise AND of
\texttt{i1} and \texttt{i2}.

\bigskip\hrule\vspace{0.1cm}
\index{bcom}
\noindent
{\bf bcom(i) : integer } \hfill {\bf bitwise complement}

\noindent
\texttt{bcom(i)} produces the bitwise complement (one's
complement) of \texttt{i}.

\bigskip\hrule\vspace{0.1cm}
\index{bit}\index{bit extraction}
\noindent
{\bf bit(i, i) : integer ? } \hfill {\bf single bit extraction}

\noindent
\texttt{bit(i,n)} returns the value of the \texttt{n}$^{th}$
bit of \texttt{i}. The indexing works the same way as strings:\\
\texttt{bit(i,1)} is the least significant bit of \texttt{i}.\\
\texttt{bit(i,0)} is the most significant bit of \texttt{i}.\\
\noindent If \texttt{n} is negative, indexing is from the most significant end,
otherwise it is from the least significant end.

\bigskip\hrule\vspace{0.1cm}
\index{bits}\index{multi-bit extraction}
\noindent
{\bf bits(i, i, i) : integer ? } \hfill {\bf multi-bit extraction}

\noindent
\texttt{bits(i,n,m)} returns the value of the \texttt{n}$^{th}$ to the
\texttt{m}$^{th}$ bit of \texttt{i}. The value is shifted down to the
least significant end of the machine word. The bits are {\em not\/} reversed if
\verb|m < n|.

\noindent \texttt{bits()} returns the number of bits in a word -- usually 32 or 64.

\bigskip\hrule\vspace{0.1cm}
\index{brot}
\noindent
{\bf brot(i, i) : integer } \hfill {\bf bitwise rotation}

\noindent
\index{bit rotation}\texttt{brot(i, j)} produces the value obtained by
rotating \texttt{i} by \texttt{j} bit positions.
If \texttt{j} is positive, the rotation is to the left;
if \texttt{j} is negative, the rotation is to the right.

\bigskip\hrule\vspace{0.1cm}
\index{bor}
\noindent
{\bf bor(i, i) : integer } \hfill {\bf bitwise or}

\noindent
\index{bitwise or}\texttt{bor(i1, i2)} produces the bitwise OR of
\texttt{i1} and \texttt{i2}.

\bigskip\hrule\vspace{0.1cm}
\index{bshift}
\noindent
{\bf bshift(i, i) : integer } \hfill {\bf signed bitwise shift}

\noindent
\index{signed shift}\texttt{bshift(i, j)} produces the value obtained by
shifting \texttt{i} by \texttt{j} bit positions.
If \texttt{j} is positive, the shift is to the left, and vacated bit
positions are filled with zeros.
If \texttt{j} is negative, the shift is to the right with sign extension.

\bigskip\hrule\vspace{0.1cm}
\index{ushift}
\noindent
{\bf ushift(i, i) : integer } \hfill {\bf unsigned bitwise shift}

\noindent
\index{unsigned shift}\texttt{ushift(i, j)} produces the value obtained by
shifting \texttt{i} by \texttt{j} bit positions.
The shift is to the left, if \texttt{j} is positive, or to the right if
\texttt{j} is negative.  Vacated bit positions are filled with zeros.

\subsection{SecureHash}
\index{SecureHash plugin}
The \texttt{SecureHash} plugin provides access to an implementation of the
RFC6324 secure hash routines. This is an extract from the description:
\begin{quote}
{\small
  This file implements the Secure Hash Algorithms
  as defined in the U.S. National Institute of Standards
  and Technology Federal Information Processing Standards
  Publication (FIPS PUB) 180-3 published in October 2008
  and formerly defined in its predecessors, FIPS PUB 180-1
  and FIP PUB 180-2.

  A combined document showing all algorithms is available at\\
  http://csrc.nist.gov/publications/fips/fips180-3/fips180-3\_final.pdf

  The five hashes are defined in these sizes:\\
  \begin{xtabular}{m{1.0in}m{3.0in}}
      \texttt{SHA-1}       &    20 byte / 160 bit\\
     \texttt{SHA-224}     &    28 byte / 224 bit\\
     \texttt{SHA-256}     &    32 byte / 256 bit\\
     \texttt{SHA-384}     &    48 byte / 384 bit\\
     \texttt{SHA-512}     &    64 byte / 512 bit\\
   \end{xtabular}
}
\end{quote}

\noindent Modified versions of the RFC6324 routines -- to make them thread-safe
-- are included in the source code of the plugin: no external libraries are
required.

Access to the hashing routines is provided at two levels
\begin{itemize}
  \item The higher (and most convenient) level is provided by the
    \texttt{SecureHash} class using the \texttt{Sha} method.
  \item At the lower level, the individual RFC6324 routines may be accessed
    directly, without using the \texttt{SecureHash} class, via the interface
    procedures provided by the plugin.
\end{itemize}

\noindent
Parameters to \texttt{Sha} may be strings, csets, open files, numbers, records or lists.\\
   String parameters are ``fed to the underlying hash routines'' ({\em fttuhr});\\
   if the parameter is a cset, each character in the set is {\em fttuhr\/};\\
   if the parameter is an open file, each line of the file is read and {\em fttuhr\/};\\
   if the parameter is a number, it is converted to a string and {\em fttuhr\/}.\\
\noindent
\texttt{Sha} applies itself recursively to list or record parameters (depth
first traversal). So records and lists may contain strings, csets, open files,
numbers, records or lists. \texttt{Sha} is tolerant of null parameters and empty
strings, csets, files or lists; they have no effect on the final hash value.

Sets and Tables are not allowed as parameters to \texttt{Sha} because their order of
enumeration is not defined. Any other type (thread, co-expression, window ...)
is also disallowed by fiat, with one exception: A procedure value%
\footnote{
A procedure value is a convenient way to steer the operation of the
\texttt{Sha} routine because it cannot be confused with any data to be added to
the hash value.}%
, using special ``command procedures'', may be used to signal that \texttt{Sha}
should take some special action as follows:
\begin{itemize}
  \item
    The procedure \texttt{More} is allowed as the final parameter. Normally the
    \texttt{Sha} method returns the secure hash of all its parameters, but if
    the final parameter is the procedure \texttt{More}, subsequent calls to
    \texttt{Sha} will continue the hashing operation.
    Note the style is
\iconcode{ Sha( ... , More ) }
\noindent rather than
\iconcode{
  Sha( ... , More() )
}
\noindent although the latter has been made to work as a ``concession to ease of use''.

\item
  The procedures \texttt{Final1} ... \texttt{Final7} may be used to signal that
  the final parameter to \texttt{Sha} is not a whole octet, only the specified
  number of bits are to be included in the hash. These procedures are only
  allowed just before the final parameter.  Where the final number of bits is
  calculated, the \texttt{Final(expr)} procedure may be
  used. \verb|0 <= expr <= 8|. (There is a \texttt{Final0} procedure, but using
  it explicitly would be slightly odd, since it will cause the final parameter
  to be ignored. There is also a \texttt{Final8} procedure which causes all of
  the final parameter to be included. These procedures are intended for use by
  the \texttt{Final} procedure).

\item
  The \texttt{Raw} procedure switches the default output from a string of
  hexadecimal characters to a string of half the length containing the actual
  bits returned by the RFC6234 result procedure. It may be placed anywhere in
  the list of parameters.
\end{itemize}

Note that \texttt{Sha(number)} will return a secure hash of the string
representation of the number, not a hash of the underlying bits. If a hash of
the bits is required, the only way to do it is to convert the number to a string
of the correct length (and endianness) without altering the value of the bits
-- Unicon strings may contain characters with all values from 0 to 255 -- this
also applies to the lower level \texttt{sha\_Input} procedure

Once \texttt{Sha} has produced a hash value (or failed) it will automatically
reset the underlying hash routines the next time it is called.

The \texttt{SecureHash} class is not thread-safe. \WarningNotThreadSafe
Using a shared \texttt{SecureHash} object in different threads without mutual
exclusion is unlikely to produce predictable results. Using a
\texttt{SecureHash} object that is private to each thread {\em is\/} safe
because the underlying hash routines are thread-safe.

The class is initialized with an optional string parameter that determines the
hash algorithm. Valid strings are \texttt{"SHA1"}, \texttt{"SHA224"},
\texttt{"SHA256"}, \texttt{"SHA384"} or \texttt{"SHA512"}.  The default is
\texttt{"SHA512"}.


The methods of the \texttt{SecureHash} class are:

\bigskip\hrule\vspace{0.1cm}
\index{Reset}
\noindent
{\bf Reset(s : "SHA512") : ? } \hfill {\bf Reset Secure Hash}

\noindent
\index{Reset()}\texttt{Reset(s)} re-initializes the secure hash instance.  The
optional parameter specifies the algorithm to use and must be one of \texttt{"SHA1"},
\texttt{"SHA224"}, \texttt{"SHA256"}, \texttt{"SHA384"} or \texttt{"SHA512"}.

\bigskip\hrule\vspace{0.1cm}
\index{Sha}
\noindent
{\bf Sha(any?, ...) : string ? } \hfill {\bf Secure Hash}

\noindent
\index{Sha()}\texttt{Sha(...)} returns the secure hash value of its
arguments, which are of the types discussed above.

\bigskip
The interface procedures of the \texttt{SecureHash} plugin are:

\bigskip\hrule\vspace{0.1cm}
\index{shaFunction}
\noindent
{\bf shaFunction(s?) : string ? } \hfill {\bf Set/Get default Hash function}

\noindent
\index{shaFunction()}
\texttt{shaFunction(h)} sets the default hash algorithm.  Valid strings are
\texttt{"SHA1"}, \texttt{"SHA224"}, \texttt{"SHA256"}, \texttt{"SHA384"} or
\texttt{"SHA512"}.  \texttt{shaFunction()} returns the name of the current
default algorithm.  The procedure is not thread-safe. \WarningNotThreadSafe

\bigskip\hrule\vspace{0.1cm}
\index{sha_Reset}
\noindent
{\bf sha\_Reset(s) : ctx ? } \hfill {\bf Initialize Secure Hash}

\noindent
\index{sha_Reset()}
\texttt{sha\_Reset(h)} initializes and returns an opaque context value that
should be fed into the other interface procedures. \texttt{h} determines the
hash algorithm. Valid strings are \texttt{"SHA1"}, \texttt{"SHA224"},
\texttt{"SHA256"}, \texttt{"SHA384"} or \texttt{"SHA512"}.

\bigskip\hrule\vspace{0.1cm}
\index{sha_Input}
\noindent
{\bf sha\_Input(ctx, s) : ? } \hfill {\bf Hash String}

\noindent
\index{sha_Input()}
\texttt{sha\_Input(x, s)} Adds the secure hash of \texttt{s} to the hash value
stored in \texttt{x}.

\bigskip\hrule\vspace{0.1cm}
\index{sha_FinalBits}
\noindent
{\bf sha\_FinalBits(ctx, c, i) : ? } \hfill {\bf Hash the final partial octet}

\noindent
\index{sha_FinalBits()}
\texttt{sha\_FinalBits(x, c, n)} Adds the secure hash of the most significant
\texttt{n} bits of \texttt{c} to the hash value stored in \texttt{x}.
\verb|1 <= n <= 7|

\bigskip\hrule\vspace{0.1cm}
\index{sha_RawResult}
\noindent
{\bf sha\_RawResult(ctx) : string ? } \hfill {\bf Return the raw hash value}

\noindent
\index{sha_RawResult()}
\texttt{sha\_RawResult(x)} returns the final hash value in the exact form
returned by the RFC6324 hash routines. The returned string will be a binary
string and may contain any character value from \texttt{char(0)} to
\texttt{char(255)}.

\bigskip\hrule\vspace{0.1cm}
\index{sha_Result}
\noindent
{\bf sha\_Result(ctx) : string ? } \hfill {\bf Return the hash value}

\noindent
\index{sha_Result()}
\texttt{sha\_Result(x)} returns the final hash value converted to a string of
hexadecimal characters.

\subsection{SQLite}
\label{SQLitePlugin}
The \texttt{SQLite} plugin provides access to version 3 of the SQLite database
engine. The SQLite software must be downloaded from
\url{https://sqlite.org/download.html} (or from a mirror or a distribution specific
site) and installed: the plugin does not provide it.

SQLite is a {\em transactional\/} database: all reads and writes take place
within a transaction -- either explicitly started by the application, or
implicitly by SQLite itself -- and a transaction is {\em atomic\/}; either all
of the modifications in the transaction succeed or none of them do.  SQLite
supports many readers but only one writer at a time. To resolve conflicts
between incompatible access requirements, requests are queued. Sometimes the
error \texttt{SQLITE\_BUSY} is returned to signal that the database is in use
and to try again later.  See \url{https://sqlite.org/transactional.html} and
\url{https://sqlite.org/lang_transaction.html} for more details.

\medskip
\index{SQLite Plugin}
The plugin gives access at three levels:
\begin{enumerate}
\item
  The highest level is a simple class, built on the routines in level 2, that
  encapsulates an SQLite database connection (and, to a large extent, handles
  \texttt{SQLITE\_BUSY}).  SQLite is transactional but SQLite transactions may
  not be nested. The class provides limited support for what {\em appears\/} to
  be nested transactions allowing calls to start and finish a transaction to be
  nested%
  \footnote{
   Methods like \texttt{SQL\_As\_List} use transactions
   internally to guarantee the consistency of the answer.
  }-- which allows routines that use a transaction to be called from other
  routines that themselves use a transaction. In reality, there is only ever one
  (the outermost) transaction.

  There is also a derived class (\texttt{RO\_SQLite})that provides a
  ``read-only'' interface, which allows reading from the database but
  prohibits its alteration.
\item
  The next level consists of utility procedures that use the lowest level routines
  to return the rows of an SQL query as a Unicon data structure (list, set,
  table). This level makes no attempt to deal with \texttt{SQLITE\_BUSY},
  passing that error back to the caller to be dealt with there.
\item
  The lowest level provides access to the SQLite API. There are almost 300
  routines in the API and, at present, this level only provides access to the
  routines that are used by the higher levels of the plugin. However, given
  the examples provided here, it is comparatively easy to extend this level to
  provide a routine that is missing.
\end{enumerate}

The thread safety of the \texttt{SQLite} plugin is a complicated question.
In summary, the external SQLite library itself {\em is\/} thread-safe but the
plugin is not; \WarningNotThreadSafe however, it can be used in a thread-safe
manner with a little extra effort.  See below (page \pageref{SQLite_ThreadSafety})
for more detail.

\subsubsection{The \texttt{SQLite} class}
\label{SQLiteClass}
The class is initialized with a string parameter that is the file name of the
database together with an optional timeout parameter -- discussed below on page
\pageref{SQLite_Timeout}  -- the default is 5 seconds.

\noindent
A simple example of the use of the \texttt{SQLite} class may be found in the
\texttt{testSQLite.icn} file that accompanies the source code of the plugin.
Using the class methods, the program creates a simple database of squares, cubes
and fourth powers of the numbers from one to ten. It then checks the values
returned by other methods of the class.

\noindent
The methods of the \texttt{SQLite} class are:
\bigskip\hrule\vspace{0.1cm}
\index{BEGIN}
\noindent
{\bf BEGIN() : ? } \hfill {\bf Start a database transaction}

\noindent
\texttt{BEGIN()} declares the start of an SQLite transaction: alterations to
the database between \texttt{BEGIN()} and \texttt{END()} will either {\em
  all\/} succeed or, none of them will succeed. Calls to \texttt{BEGIN} may
be nested for convenience but note that SQLite does not support nested
transactions: from the point of view of the database software there is only
ever one transaction -- the nested transactions are a fiction provided by the
\texttt{SQLite} class so that functions which, in isolation, call for a
transaction may be conveniently amalgamated into a single overall transaction.
After a successful call of \texttt{BEGIN} the \texttt{SQLITE\_BUSY} status will
not be returned by any SQLite routine up to (but not including) the
corresponding call of \texttt{END}.

\bigskip\hrule\vspace{0.1cm}
\index{END}
\noindent
{\bf END() : ? } \hfill {\bf Complete a database transaction}

\noindent
\texttt{END()} declares the end of an SQLite transaction. If the call
succeeds, all modifications to the database after the call of the
corresponding call to \texttt{BEGIN} will have succeeded.

\bigskip\hrule\vspace{0.1cm}
\index{ROLLBACK}
\noindent
{\bf ROLLBACK() : ? } \hfill {\bf Abandon a database transaction}

\noindent
\texttt{ROLLBACK()} declares the end of an SQLite transaction. If the call
succeeds, all modifications to the database after the call of the
outermost call to \texttt{BEGIN} will have been undone and the state of
the database will be as it was just before the call to \texttt{BEGIN}.

\bigskip\hrule\vspace{0.1cm}
\index{Close}
\noindent
{\bf Close() : } \hfill {\bf Close a database}

\noindent
\texttt{Close()} closes a database connection and recovers resources. If
\texttt{Close} is called during a transaction, the transaction is automatically
rolled back. It is important to call \texttt{Close} before exiting the program:
do not rely on the Unicon system to close the connection for you.

\bigskip\hrule\vspace{0.1cm}
\index{Exec}
\noindent
{\bf Exec(s, ...) : row|val? } \hfill {\bf Execute SQL}

\noindent
\texttt{Exec(sql)} prepares the SQL query in \texttt{sql} and then uses the
SQLite library to execute it. The query should return, at most, one row of data.
If the query asks for more than one value the values will be returned in a
record whose field names are the names used in the query. A query that asks for a
single value results in that value being returned directly (rather than a record
with a single field). A query that results in no data being returned will either
succeed or fail.  \texttt{Exec(sql, p1, p2 ...)}  will bind the parameters to
the query before execution.

\bigskip\hrule\vspace{0.1cm}
\index{thing}
\noindent
{\bf ErrMsg() : string } \hfill {\bf Return the most recent error message}

\noindent
\texttt{ErrMsg()} returns a string (in English) that describes the most recent
error detected by the SQLite database routines.

\bigskip\hrule\vspace{0.1cm}
\index{isTable}
\noindent
{\bf isTable(s) :  ? } \hfill {\bf Test if a SQL table exists}

\noindent
\texttt{isTable(t)} succeeds if the SQL table \texttt{t} exists in the database.

\bigskip\hrule\vspace{0.1cm}
\index{Rows}
\noindent
{\bf Rows(s) : integer ? } \hfill {\bf Count rows in a SQL table}

\noindent
\texttt{Rows(t)} returns the number of rows in the SQL table \texttt{t}.

\bigskip\hrule\vspace{0.1cm}
\index{SQL_Row}
\noindent
{\bf SQL\_Row(s, ... ) : row * } \hfill {\bf Get data from a query}

\noindent
\texttt{SQL\_Row(sql, ...)} prepares the query in \texttt{sql} and then returns
the data one row at a time. The data is in the same form as \texttt{Exec} -- a
record with named fields or a single value.  Note that calling \texttt{SQL\_Row}
fewer times than there are available rows will leak memory until the database
connection is closed.

\bigskip\hrule\vspace{0.1cm}
\index{SQL_As_List}
\noindent
{\bf SQL\_As\_List(s, ...) : list ? } \hfill {\bf Return query data as a list}

\noindent
\texttt{SQL\_As\_List(sql, ...)}  prepares the query in \texttt{sql} and returns
the results in a list. The list elements will either be records with named
fields or single values (same format as \texttt{Exec}).

\bigskip\hrule\vspace{0.1cm}
\index{SQL_As_Set}
\noindent
{\bf SQL\_As\_Set(s, ...) : set ? } \hfill {\bf Return query data as a set}

\noindent
\texttt{SQL\_As\_Set(sql, ...)}  prepares the query in \texttt{sql} and returns
the results in a set. The set elements will either be records with named
fields or single values (same format as \texttt{Exec}).
Note that queries that return a single value will have duplicates removed but
queries that return a record won't because all records are unique, even if the
fields have identical values.

\bigskip\hrule\vspace{0.1cm}
\index{SQL_As_Table}
\noindent
{\bf SQL\_As\_Table(s, i : 1, ...) : table ? } \hfill {\bf Return query data as a table}

\noindent
\texttt{SQL\_As\_Table(sql, n, ...)}  prepares the query in \texttt{sql} and
returns the results in a Unicon table whose keys are taken from column \texttt{n} of
the query data. The table element values will always be records, even if the
data rows have only one value (and will include the indexing column value). If
the indexing column has duplicate values then later rows will overwrite earlier
rows with the same key.

\subsubsection{Thread safety of the \texttt{SQLite} class}
\label{SQLite_ThreadSafety}
The main reason that the \texttt{SQLite} class is not thread-safe is down to how
\WarningNotThreadSafe it handles transactions. SQLite itself supports multiple
simultaneous read transactions coming from separate database connections,
possibly in separate threads or processes, but only one simultaneous write
transaction. Each instance of the \texttt{SQLite} class counts how many
\texttt{BEGIN} methods are active on its connection and only isues a
\texttt{BEGIN IMMEDIATE TRANSACTION} statement for the outermost (first) call
and only issues a \texttt{COMMIT} statement when the outermost \texttt{END}
method is called.  If the instance is shared between threads then it is easy for
this count to become confused, leading to a \texttt{COMMIT} statement being
executed at an inappropriate point.

To avoid this, use a separate instance of the class in each thread. Doing so
avoids most (but not all) problems. If concurrent access is made to the database
from separate threads then it is possible that the \texttt{SQLITE\_BUSY} error
is returned.  The \texttt{BEGIN} and \texttt{END} methods handle this by
delaying for a short time and retrying but they will fail if the total time
taken is more than the wait time specified in the initialization of the
\label{SQLite_Timeout}
class. Hence a program that uses concurrent access and has transactions that
might take a long time must be prepared to deal with the failure.

An alternative design which avoids these problems is to use a single database
access thread to serialize all SQL queries and to use Unicon's message passing
facilities to send queries and receive answers.  If there is a mixture of read
and write requests the reduction in parallelism (compared with one class
instance per thread) is not as drastic as it appears because the reduction in
parallel queries has {\em already\/} occurred due to the way the \texttt{SQLite}
class uses transactions -- each method uses \texttt{BEGIN} and \texttt{END}
internally to make sure that methods like \texttt{SQL\_As\_List} return a
consistent set of data and don't fail in the middle of constructing the list.
However, if most of the queries are read only, this design will reduce the number
of concurrent queries dramatically (to one).

There is, in essence, an unfortunate trade-off between performance and
convenience: for maximum throughput the only way is to eschew the
\texttt{SQLite} class, use the utility procedures or the low level interface
routines explicitly, do as much as possible in parallel and deal with
\texttt{SQLITE\_BUSY} wherever it occcurs.  In practice, the \texttt{SQLite}
class performs tolerably well and it is only those applications that demand the
best performance and the highest possible level of parallel access to the
database that may feel the need to replace it. In those circumstances, it
might also be fruitful to reconsider the use of an interpreted language.

\subsubsection{The \texttt{SQLite} utility procedures}
There is a deliberate similarity in names between the methods of the
\texttt{SQLite} class and the utility procedures used by them. The functionality
is largely the same with the significant exception that the utility procedures
make no attempt to handle \texttt{SQLITE\_BUSY} -- that error is passed back
to the caller to be dealt with there. In many cases, the class method just
wraps the utility procedure it uses in \texttt{BEGIN} \ldots \texttt{END} calls.

The utility procedures of the \texttt{SQLite} plugin are:
\bigskip\hrule\vspace{0.1cm}
\index{SQLi_isTable}
\noindent
{\bf SQLi\_isTable(s) :  ? } \hfill {\bf Test if a SQL table exists}

\noindent
\texttt{SQLi\_isTable(t)} succeeds if the SQL table \texttt{t} exists in the database.

\bigskip\hrule\vspace{0.1cm}
\index{SQLi_Rows}
\noindent
{\bf SQLi\_Rows(s) : integer ? } \hfill {\bf Count rows in a SQL table}

\noindent
\texttt{SQLi\_Rows(t)} returns the number of rows in the SQL table \texttt{t}.

\bigskip\hrule\vspace{0.1cm}
\index{SQLi_Exec}
\noindent
{\bf SQLi\_Exec(s, ...) : row|val? } \hfill {\bf Execute SQL}

\noindent
\texttt{SQLi\_Exec(sql)} prepares the SQL query in \texttt{sql} and then uses
the SQLite library to execute it. The query should return, at most, one row of
data.  If the query asks for more than one value the values will be returned in
a record whose field names are the names used in the query. A query that asks for a
single value results in that value being returned directly (rather than a record
with a single field). A query that results in no data being returned will either
succeed or fail.  \texttt{SQLi\_Exec(sql, p1, p2 ...)}  will bind the parameters
to the query before execution.

\bigskip\hrule\vspace{0.1cm}
\index{SQLi_Row}
\noindent
{\bf SQLi\_Row(s, ... ) : row * } \hfill {\bf Get data from a query}

\noindent
\texttt{SQLi\_Row(sql, ...)} prepares the query in \texttt{sql} and then returns
the data one row at a time. The data is in the same form as \texttt{SQLi\_Exec} -- a
record with named fields or a single value.  Note that calling \texttt{SQLi\_Row}
fewer times than there are available rows will leak memory until the database
connection is closed.

\bigskip\hrule\vspace{0.1cm}
\index{SQLi_As_List}
\noindent
{\bf SQLi\_As\_List(s, ...) : list ? } \hfill {\bf Return query data as a list}

\noindent
\texttt{SQLi\_As\_List(sql, ...)}  prepares the query in \texttt{sql} and returns
the results in a list. The list elements will either be records with named
fields or single values (same format as \texttt{SQLi\_Exec}).

\bigskip\hrule\vspace{0.1cm}
\index{SQLi_As_Set}
\noindent
{\bf SQLi\_As\_Set(s, ...) : set ? } \hfill {\bf Return query data as a set}

\noindent
\texttt{SQLi\_As\_Set(sql, ...)}  prepares the query in \texttt{sql} and returns
the results in a set. The set elements will either be records with named
fields or single values (same format as \texttt{SQLi\_Exec}).
Note that queries that return a single value will have duplicates removed but
queries that return a record won't because all records are unique, even if the
fields have identical values.

\bigskip\hrule\vspace{0.1cm}
\index{SQLi_As_Table}
\noindent
{\bf SQLi\_As\_Table(s, i : 1, ...) : table ? } \hfill {\bf Return query data as a table}

\noindent
\texttt{SQLi\_As\_Table(sql, n, ...)}  prepares the query in \texttt{sql} and
returns the results in a Unicon table whose keys are taken from column
\texttt{n} of the query data. The table element values will always be records,
even if the data rows have only one value (and will include the indexing column
value). If the indexing column has duplicate values then later rows will
overwrite earlier rows with the same key.
\subsubsection{The \texttt{SQLite} interface routines}

The plugin uses a small subset of the three hundred or so routines in the SQLite
API. There is comprehensive documentation on each of the individual routines
provided by the \texttt{SQLite} library at
\url{https://www.sqlite.org/docs.html} and no attempt is made to reproduce that
material here.  Each of the interface procedures provided by the plugin checks
that the supplied parameters are of the correct type, converts them from their
Unicon representation into something that conforms to the \texttt{C} calling
convention and passes them down to the underlying SQLite library routine.

It is anticpated that most users will not call these routines directly; instead,
preferring to use the \texttt{SQLite} class interface or the higher level
utility procedures of the plugin.

\medskip
\noindent
\tablehead{%
  \hline
  Interface procedure & SQlite routine & Notes\\
  \hline
}%
\tabletail{%
  \hline
  \multicolumn{3}{|r|}{\small continued \ldots}\\
  \hline
}%
\tablelasttail{\hline}%
\begin{xtabular}{|>{\small\texttt\bgroup}m{1.8in}<{\egroup}%
    >{\small\texttt\bgroup}m{1.5in}<{\egroup}%
    |>{\small\bgroup}p{3.2in}<{\egroup}|}
  SQLi\_Init(x?) : & sqlite3\_config &
  This procedure should be called before any other interface routine. If the
  parameter is null, column numbers will start at one. If non-null they will
  start at zero. The higher level routines and the \texttt{SQlite} class
  assume that the column numbers start at one.\smallskip\\
  %
  \hline
  SQLi\_libversion() : s & sqlite3\_libversion &
  Returns the version of the SQLite library as a string.\\
  %
  SQli\_libversion\_number() : i & sqlite3\_libversion\_number &
  Returns the version of the SQLite library as an integer.\\
  %
  \hline
  SQLi\_open(s, s?) : ctx? & sqlite3\_open\_V2 &
  Open (or create) the database named by the first parameter. The second
  parameter is \texttt{"b"} for read/write access and anything else for
  read-only access.\\
  %
  \hline
  SQLi\_close(ctx) : & sqlite3\_close\_V2 &
  Close the database connection.
  It is important to call \texttt{SQLi\_close} before exiting the program:
  do not rely on the Unicon system to close the connection for you.\\
   %
  \hline
  SQLi\_prepare(ctx, s) :sqst? & sqlite3\_prepare\_V2 &
  Compile (prepare for execution) the SQL statement \texttt{s}.\\
  %
  \hline
  SQLi\_bindArg(sqst, i, x) : ? &
  \begin{tabular}{l}
    sqlite3\_bind\_null\\
    sqlite3\_bind\_int64\\
    sqlite3\_bind\_double\\
    sqlite3\_bind\_text\\
  \end{tabular}
  &
  Bind a single parameter \texttt{x} to the column specified by
  \texttt{i}. \texttt{x} may be null, integer, real or a string.\\
  %
  \hline
  SQLi\_bind(sqst, x, ...) : ? &
  \begin{tabular}{l}
    sqlite3\_bind\_null\\
    sqlite3\_bind\_int64\\
    sqlite3\_bind\_double\\
    sqlite3\_bind\_text\\
  \end{tabular}
  &
  Bind parameters, starting at the first column.
  Each parameter may be null, integer, real or a string.\\
  %
  \hline
  SQLi\_step(sqst, i: 0) : ?  & sqlite3\_step &
  Evaluate (execute) the prepared statement. A status return of
  \texttt{SQLITE\_BUSY} will cause the program to stop unless \texttt{i} is non
  zero.\\
  %
  \hline
  SQLi\_errmsg(ctx) : s & sqlite3\_errmsg &
  Return the most recent error message.\\
  %
  %\hline
  SQLi\_Error(i) : s & sqlite3\_errstr &
  \texttt{SQLi\_Error(n)} returns the error message associated with the return
  status \texttt{n}.\\
  %
  \hline
  SQLi\_column\_count(sqst) : i & sqlite3\_column\_count &
  Return the number of columns in the result set returned by the prepared statement.\\
  %
  \hline
  SQLi\_column\_type(sqst, i) : i & sqlite3\_column\_type &
  Return the data type code of the specified column in the result set returned by
  the prepared statement.\\
  %
  \hline
  SQLi\_column\_name(sqst, i) : s & sqlite3\_column\_name &
  Return the assigned name of the specified column in the result set returned by
  the prepared statement.\\
  %
  \hline
  SQLi\_column\_string(sqst, i) : s & sqlite3\_column\_text &
  Return the value of the specified column as a string.\\
  %
  \hline
  SQLi\_column\_integer(sqst, i) : i & sqlite3\_column\_int64 &
  Return the value of the specified column as an integer.\\
  %
  \hline
  SQLi\_column\_real(sqst, i) : r & sqlite3\_column\_double &
  Return the value of the specified column as a real number.\\
  %
  \hline
  SQLi\_column(sqst, i) : x &
    \begin{tabular}{l}
    sqlite3\_column\_type\\
    sqlite3\_column\_int64\\
    sqlite3\_column\_double\\
    sqlite3\_column\_text\\
    \end{tabular} &
    Return the value of the specified column as specified by its data type code.\\
  %
  \hline
  SQLi\_finalize(sqst) & sqlite3\_finalize &
  Recover resources from a prepared SQL statement after execution.\\
  \multicolumn{3}{|p{6.5in}|}{\small
    Although the SQLite documentation is generally not repeated here, it is
    worth emphasizing this extract:
    \begin{quote}
    The application must finalize every prepared statement in order to avoid
    resource leaks. It is a grievous error for the application to try to use a
    prepared statement after it has been finalized. Any use of a prepared
    statement after it has been finalized can result in undefined and
    undesirable behavior such as segfaults and heap corruption.
    \end{quote}
  }\\
  \hline
\end{xtabular}
% Be considerate of other uses of xtabular later on in the book that might not
% expect these macros to have anything in them.
\tablefirsthead{}%
\tablehead{}%
\tabletail{}%
\tablelasttail{}%
%
%================================================================================
%
% This material (from PR#329, which was withdrawn) is preserved here in the hope
% that it will be useful if the SQLite plugin makes the transition from an
% experimental addition into the main part of the book.
%
% N.B. At that time, the material may require revision.
%
% (1) ---------- Replace these lines in 6.3 Database Architectures and ODBC
% - \index{ODBC}SQL databases are accessed through an underlying Open
% - DataBase Connectivity (ODBC) transport mechanism. This mechanism allows
% ---------- with the following
%% \index{ODBC}
%% \index{SQLite}
%% Unicon provides two complementary ways to access an SQL database: via the
%% Open Database Connectivity mechanism (ODBC) or using the the SQLite
%% plugin. There is no choice if an existing database is being accessed --
%% the SQLite plugin cannot be used to access an ODBC database and the ODBC
%% mechanism cannot access an SQLite database. But if the database is being
%% created from scratch, some thought should be given to which option to
%% use. This extract (from \texttt{https://sqlite.org}) summarizes the
%% different approaches.
%% \begin{quote}
%%  {\sl
%%  SQLite is not directly comparable to client/server SQL database engines
%%  such as MySQL, Oracle, PostgreSQL, or SQL Server since SQLite is trying
%%  to solve a different problem.
%%
%%  Client/server SQL database engines strive to implement a shared
%%  repository of enterprise data. They emphasize scalability, concurrency,
%%  centralization, and control. SQLite strives to provide local data
%%  storage for individual applications and devices. SQLite emphasizes
%%  economy, efficiency, reliability, independence, and simplicity.
%%  }
%% \end{quote}
%% In essence, SQLite is aimed more at replacing a system using conventional files
%% for persistent storage than competing with an enterprise level database.
%%
%% An ODBC database is more difficult to set up, but may be used from
%% several machines simultaneously. Apart from the SQLite library itself,
%% an SQLite database needs no extra software to set it up but is hosted
%% on a single machine (although access to it can be multiplexed using
%% Unicon's networking features); it may also reside in memory, providing
%% the advantages of a memory based database with the power and
%% flexibility of SQL.
%%
%% An ODBC database may be accessed using Unicon's standard functions, which
%% work in the same way on DBM and SQL databases, making migration
%% particularly easy. The SQLite plugin only provides access via SQL which
%% means that more of the SQL language must be mastered by the programmer in
%% order to get a working application. The advantage (compared to the ODBC
%% based method) is that the {\em full\/} SQL language may be used: for example,
%% the workaround discussed on page \pageref{NestedSQL} to use nested
%% \texttt{SELECT} queries is not required.
%%
%% Accessing an SQL database through an underlying Open
%% DataBase Connectivity transport mechanism allows
%%---------- End of replacement text ----------
%
% (2) ---------- Addition to the end of 6.4 SQL Databases ----------
%% \subsection*{Opening an SQLite database}
%% The ODBC infrastructure described above is not neccessary when using SQLite.
%% To connect to or create an SQLite database, declare an instance of the
%% \texttt{SQLite} class with a parameter of the filename that contains the
%% database.
%%
%% \iconcode{
%%   db := SQLite("myDatabase.sqlite")
%% }
%% \noindent
%% Object oriented programming, together with Unicon's classes, is discussed
%% later in part two of this book. For now, \texttt{db} is a variable that
%% can be used to access the {\em methods\/} of the class that in turn allow
%% read and write access to the database using SQL. In contrast to an ODBC
%% database, no username or password is needed.  If the given filename does
%% not exist, an empty SQLite database will be created.
%%
%% The \texttt{SQLite} class is described in appendix C of this book on page
%% \pageref{SQLiteClass}.
%%
%% \subsection*{Querying and modifying a SQLite database}
%%
%% The \texttt{SQLite} class has several methods for querying and modifying
%% the database. They all have a first parameter which is the SQL statement
%% to execute, followed by optional extra parameters. The simplest is
%% \texttt{Exec} which is designed for SQL queries that return a single row
%% of data or, perhaps, no data at all.
%% \iconcode{
%%   people := db.Exec("SELECT count(*) FROM employees")
%% }
%% \noindent
%% will return the number of rows in the employees table (in this case no
%% extra parameters are needed). A simple example of a query that uses values
%% from the calling program would be
%% \iconcode{
%%   people := db.Exec("SELECT count(*) FROM jobs WHERE payrate > ?", x)
%% }
%% \noindent
%% which returns the number of people who are paid more than \texttt{x}.
%% The \texttt{?} acts as a place holder for the value to be used.
%% Each \texttt{?} in the query refers to the next parameter. To use a
%% parameter more than once, append the positional value of the parameter
%% (e.g. use \texttt{?3} to refer to the third parameter in the argument
%% list).  There are some traps when using positional parameters and it is
%% best to either not use them at all in a query or make sure that every
%% placeholder has an explicit positional value.
%%
%% To print out the name and job title of every employee
%% \iconcode{
%%   every e := db.SQL\_Row("SELECT name, title FROM employees \_ \\
%%   \>\> JOIN jobs ON employees.id = jobs.id") do write(e.name, " ", e.title)
%% }
%% \noindent
%% The \texttt{SQL\_Row} method returns the value of each row returned by the
%% query as a record whose fields are the names used in the SQL query.
%% The \texttt{SQLite} class also has methods that return all of the rows of
%% a query at once in the appropriate Unicon type:
%% \texttt{SQL\_As\_List} returns the rows in a list.
%% \texttt{SQL\_As\_Table} returns them in a (Unicon) table and
%% \texttt{SQL\_As\_Set} returns them in a set.
%%
%% To add new employees (assuming they are in a list of records called
%% \texttt{newcomers})
%% \iconcode{
%% db.BEGIN()\\
%% \>    id := 1 + db.EXEC("SELECT max(id) FROM employees")\\
%% \>    every  newbie := !newcomers do \{\\
%% \>\>    db.Exec("INSERT INTO employees (id, name, phone) VALUES(?,?,?)",\\
%% \>\>\>id, newbie.name, newbie.phone)\\
%% \>\>    db.Exec(INSERT INTO jobs (id, payrate, is\_salaried, title)  VALUES(?,?,?,?)",\\
%% \>\>\>id, newbie.pay, newbie.salaried, newbie.jobtitle)\\
%% \>\>  id+:=1\\
%% \>\}\\
%% db.END()
%% }
%% \noindent
%% Using \texttt{BEGIN} and \texttt{END} ensures that either all of the
%% updates are performed or (in case of a failure) none of them. It will also
%% ensure that any other application using the database will pause until the
%% transaction is complete, so an inconsistent view of the data will be
%% avoided.  See appendix C for more details.

%% \subsection*{Closing an SQLite database}
%% It is easy to develop the habit of relying on the Unicon system to close
%% any files that have been left open on program termination, even though it
%% is not recommended to do so. {\em This will not work\/} on SQLite
%% databases because the connection to the database is not a Unicon
%% file. All connections should be closed explicitly by using the
%% \texttt{Close} method.
%%
%----------------------------------------------------------------------

