\chapter{Advanced Language Features}

The previous chapters described a wide range of built-in computational
facilities that help to make Unicon a great language.
This chapter delves into interesting features that make Unicon
more than just the sum of its parts. This chapter demonstrates:

\begin{itemize}\itemsep0pt
\item Controlling expressions more precisely
\item Using list structures and procedure parameter lists
interchangeably
\item Holding a \index{generator}generator expression in a value so that
its results can be used in different locations throughout the program
\item Defining your own \index{control structure}control structures
\item Evaluating several generator expressions in parallel
\item Permuting strings using sophisticated mappings
\item Using a more efficient list representation
\end{itemize}

\section{Limiting or Negating an Expression}

\index{limiting an expression}Chapter 1 described generators and the
expression mechanism without mentioning many methods for using them,
other than \texttt{every} loops. Suppose you wish to generate five
elements from a table. If the table has thousands of elements, then you
may want to generate just five elements precisely in a situation where
generating all the table elements with \texttt{!T} is infeasible. You
could write an \texttt{every} loop that breaks out after five
iterations, but this solution isn't easy to use within
some more complex expressions. The binary backslash operator
\texttt{\textit{expr}}\texttt{ {\textbackslash} }\texttt{\textit{i}}
limits \texttt{\textit{expr}} to at most \texttt{\textit{i}} results.
If \texttt{expr} has fewer results, the limitation operator has no
effect; once \texttt{\textit{i}} results have been obtained, limitation
causes the expression to \index{expression failure}fail even if it
could produce more results.

Unicon does not have a boolean type, so
Chapter 1 downplayed the standard logical operators. The
\index{alternation operator ( {\textbar} )}alternation operator
(\texttt{{\textbar}}) resembles a short-circuit \index{OR operator}OR
operator, since it generates its left operand and only evaluates its
right operand if the left operand or the surrounding expression fails.
The conjunction operator (\texttt{\&}) resembles a short-circuit
\index{AND operator}AND operator; it evaluates its left operand,
and if that operand succeeds, the result of the conjunction is the
result of its right operand. The reserved word \texttt{not} rounds out
the boolean-like operators. If \texttt{\textit{expr}} produces no
results, then \index{not}\texttt{not }\texttt{\textit{expr}} will
succeed (and produce a null value); if \texttt{\textit{expr}} produces
any results, then the \texttt{not} expression fails. The \texttt{not}
operator can remedy certain forms of generator confusion. Compare the
following two expressions:

\iconcode{
if not (s ==
("good"{\textbar}"will"{\textbar}"hunting"))
then write("nope") \\
if (s \~{}==
("good"{\textbar}"will"{\textbar}"hunting"))
then write("uh huh")}

The first expression uses \texttt{not} to ensure that string \texttt{s}
is none of the three words. The second expression always writes
\texttt{"uh huh"}, because any string
\texttt{s} that you pick will be not equal (\texttt{\~{}==}) to at
least one of the three strings in the alternation. The \texttt{then}
part will always execute, which is probably not what was intended.

{\sffamily\bfseries Note}

{\sffamily
Negating an == operator is not the same as using a \~{}== operator!}

The \index{conjunction \&}conjunction operator
\texttt{\textit{expr}}\texttt{\textit{\textsubscript{1}}}\texttt{ \&
}\texttt{\textit{expr}}\texttt{\textit{\textsubscript{2}}} has an
alternate syntax, a comma-separated list of expressions in parentheses:
\texttt{(}\texttt{\textit{expr}}\texttt{\textit{\textsubscript{1}}}\texttt{
, }\texttt{\textit{expr}}\texttt{\textit{\textsubscript{2}}}\texttt{)}.
Any number of expressions may be present; the whole expression
succeeds if they all succeed. This looks similar to procedure call syntax
because it is similar: a procedure call mutually
evaluates all the actual parameters before the procedure is invoked.
Invocation allows a string or integer in the ``procedure'' slot
in front of a parenthesized argument
list. For a string, as in
\texttt{s(x)}, a procedure by the name given in \texttt{s} is called;
if \texttt{s} had the value \texttt{"foo"},
then \texttt{s(x)} is the same as \texttt{foo(x)}. For an integer value
\texttt{i}, after all arguments are evaluated, the value of the entire
expression is the value of the \texttt{i}'th argument.

\section{List Structures and Parameter Lists}

The functions \texttt{write()} and \texttt{put()} take any number of
arguments; this flexibility is powerful and convenient.
You can write \index{variable!no. of arguments}variable argument
procedures of your own by ending the last parameter in your procedure
declaration with empty square brackets:

\iconcode{
procedure myfunc(x, y, z[])}

\noindent
In this case, instead of throwing away all arguments after the third,
the third parameter and all parameters that follow are placed into a
newly-constructed list. A call to the above procedure with
\texttt{myfunc(1, 2, 3, 4, 5)} causes \texttt{z} to have the value
\texttt{[3, 4, 5]}.

It is also useful to do the opposite and construct a list
of dynamic (or user-supplied) length, and then call a
procedure with that list as its parameter list. The
\index{list!invocation}\index{apply operator}apply operator, binary
\texttt{!} performs this feat. If you call \texttt{write ! L}, then all
the elements of \texttt{L} are written contiguously on a single line
(unless they contain newline characters).

\section{Co-expressions}

A \index{co-expression}co-expression is an independent, encapsulated
\index{thread}thread{}-like context, where the results of an expression
(hopefully a generator!) can be picked off one at a time.
Suppose you are writing a program that generates
code, and you need something that will generate unique variable names.
This expression will generate names:

\iconcode{
"name" {\textbar}{\textbar} seq()}

\noindent
Function \index{seq()}\texttt{seq()} produces a progressive sequence
of integers, by default starting at 1, so the whole expression
generates \texttt{"name1"},
\texttt{"name2"},
\texttt{"name3"}, ... You can
use this expression anywhere in your code; but you may need
it in several different places.

There are times when you need to separate the {\em evaluation\/} of an
expression from its location in the program. The normal
way to do this would be a procedure. You can make separate calls
to a procedure from different locations in your program, but there is
no easy way to use the results from a single \index{instance}instance
of a generator in multiple locations. You can put all the results in a
list (not a good idea for generators with infinite result sequences) or
rewrite the procedure to produce the sequence using separate calls, but
this requires static or global variables, and is awkward at best:

\iconcode{
procedure nameseq() \\
static i \\
initial i := 0 \\
\>   return "name" {\textbar}{\textbar} (i+:= 1) \\
end
}

Now, consider the code generating program again. It may need not one
name sequence, but two kinds of names: statement labels and
temporary variables. It would be poor engineering to write a different
procedure for each such sequence. The \texttt{nameseq()}
procedure was already cumbersome for so simple a task, but generalizing
it for multiple kinds of names makes it \textit{really} messy. By
creating a pair of co-expressions, you can capture exactly what is
needed with a lot less code:

\iconcode{
labelname := create ("\_L"
{\textbar}{\textbar} seq()) \\
varname := create("\_V" {\textbar}{\textbar}
seq())
}

\noindent
In both cases, \index{create}\texttt{create}\texttt{
}\texttt{\textit{expr}} allocates and initializes an evaluation context
plus the memory needed to evaluate expression \textit{expr}, but does
not start to evaluate it. Since the co-expression value may be used
outside the procedure call where it is created, the evaluation context
includes a copy of the local variables and parameters used in the
expression. When a co-expression is \textit{activated}, it produces the
next value. A co-expression is activated by the \texttt{@} operator. Each
activation of \texttt{labelname} produces the next string in the
sequence \texttt{"\_L0"},
\texttt{"\_L1"},
\texttt{"\_L2"}, and so on. Similarly, each
activation \texttt{@varname}\texttt{ }produces the next in the sequence
\texttt{"\_V0"},
\texttt{"\_V1"},
\texttt{"\_V2"}, and so on. 

\iconcode{
loop\_name := @labelname \\
tempvar\_name := @varname
}

After a co-expression has produced all its results, further evaluation
with \texttt{@} will \index{fail!co-expression}fail. The \texttt{\^{}}
operator produces a new co-expression with the same expression as its
argument, but "rewound" to the beginning.

\iconcode{
\>   c := \^{}c}

\section{User-Defined Control Structures}

\index{control structure}\textit{Control structures} are language elements
that determine in what order, and how many times, expressions
are executed. Co-expressions are used to implement new
control structures when procedures that take
co-expression parameters control the order and number
of times they are activated.
Consider a control structure that selects values from the first
expression at positions specified by the second. This could be
called as:

\iconcode{
seqsel([create fibonacci(), create primes()])}

Assuming that you have a pair of \index{generator}generator procedures
that produce the Fibonacci numbers (1, 1, 2, 3, 5, 8, 13, \ ?) and the
primes (2, 3, 5, 7, 11, ?), this expression produces the numbers 1, 2,
5, 13, 89, .... Here is the implementation of \texttt{seqsel()}:

\iconcode{
procedure seqsel(a) \\
\>   (*a = 2) {\textbar} stop("seqsel requires a list of
two arguments") \\
\>   e1 := a[1]; e2 := a[2] \\
\ \ \ \# position in the first stream we are looking at \\
\>   index := 1 \\
\>   repeat \{ \\
\>   \ \ \ \# Get the next index \\
\>   \ \ \ (i := @e2 ) {\textbar} fail \\
\ \ \ \ \ \ \# Keep getting values from the second expression until \\
\ \ \ \ \ \ \# we get to the i'th one. If e1 cannot produce that \\
\ \ \ \ \ \ \# many values, we fail. \\
\>   \ \ \ every index to i do \\
\>\>   \ \ \ (value := @e1) {\textbar} fail \\
\>   \ \ \ suspend value \\
\>   \ \ \ index := i+1 \\
\>   \ \ \ \} \\
end
}

\noindent
Unicon provides a syntactic short-cut for this kind of usage:

\iconcode{
proc([create e1, create e2, ..., create en])
}

\noindent
can also be written with curly brackets, as

\iconcode{
proc\{e1, e2, ..., en\}
}

\section{Parallel Evaluation}

\index{parallel evaluation}Co-expressions can be used to evaluate
expressions "in parallel".
This program writes a table of \index{ASCII}ASCII characters with the
hex, decimal, and octal equivalents:

\iconcode{
procedure main() \\
\>   dec := create(0 to 255) \\
\>   hex\_dig := "0123456789abcdef" \\
\>   hex := create(!hex\_dig {\textbar}{\textbar} !hex\_dig) \\
\>   oct := create((0 to 3) {\textbar}{\textbar} (0 to 7)
{\textbar}{\textbar} (0 to 7)) \\
\>   char := create image(!\&cset) \\
\>   while write(@dec, "{\textbackslash}t",
@oct, "{\textbackslash}t", @hex,
"{\textbackslash}t", @char) \\
end
}

Co-expression \texttt{dec} produces the sequence 0, 1, 2, ... 255;
\texttt{hex} the sequence \texttt{"00"},
\texttt{"01"},
\texttt{"03"}, ...
\texttt{"ff"}; \texttt{oct}\texttt{ }the
sequence \texttt{"001"},
\texttt{"002"}, ...
\texttt{"377"}; and \texttt{char}\texttt{
}the sequence ..., \texttt{" "},
\texttt{"!"}, ...,
\texttt{"A"}, ...
\texttt{"Z"}, ...,
\texttt{"a"}, ...
\texttt{"z"}, and so forth.

Every invocation of \texttt{write()} results in all the co-expressions
being activated once, so they are all run in lock-step, producing this
table:

{\sffamily
0 \ \ \ \ \ \ 000 \ \ \ \ 00
\ \ \ \ \ "{\textbackslash}x00"}

{\sffamily
1 \ \ \ \ \ \ 001 \ \ \ \ 01
\ \ \ \ \ "{\textbackslash}x01"}

{\sffamily
2 \ \ \ \ \ \ 002 \ \ \ \ 02
\ \ \ \ \ "{\textbackslash}x02"}

{\sffamily
...}

{\sffamily
45 \ \ \ \ \ 055 \ \ \ \ 2d \ \ \ \ \ "-"}

{\sffamily
46 \ \ \ \ \ 056 \ \ \ \ 2e \ \ \ \ \ "."}

{\sffamily
47 \ \ \ \ \ 057 \ \ \ \ 2f \ \ \ \ \ "/"}

{\sffamily
48 \ \ \ \ \ 060 \ \ \ \ 30 \ \ \ \ \ "0"}

{\sffamily
49 \ \ \ \ \ 061 \ \ \ \ 31 \ \ \ \ \ "1"}

{\sffamily
50 \ \ \ \ \ 062 \ \ \ \ 32 \ \ \ \ \ "2"}

{\sffamily
...}

{\sffamily
90 \ \ \ \ \ 132 \ \ \ \ 5a \ \ \ \ \ "Z"}

{\sffamily
91 \ \ \ \ \ 133 \ \ \ \ 5b \ \ \ \ \ "["}

{\sffamily
92 \ \ \ \ \ 134 \ \ \ \ 5c
\ \ \ \ \ "{\textbackslash}{\textbackslash}"}

{\sffamily
93 \ \ \ \ \ 135 \ \ \ \ 5d \ \ \ \ \ "]"}

{\sffamily
94 \ \ \ \ \ 136 \ \ \ \ 5e
\ \ \ \ \ "\^{}"}

{\sffamily
95 \ \ \ \ \ 137 \ \ \ \ 5f \ \ \ \ \ "\_"}

{\sffamily
96 \ \ \ \ \ 140 \ \ \ \ 60
\ \ \ \ \ "'"}

{\sffamily
97 \ \ \ \ \ 141 \ \ \ \ 61 \ \ \ \ \ "a"}

{\sffamily
...}

{\sffamily
255 \ \ \ \ 377 \ \ \ \ ff
\ \ \ \ \ "{\textbackslash}xff"}

\noindent
Parallel evaluation can also be used to assign to a set of variables:

\iconcode{
ce := create !stat(f) \\
every (dev {\textbar} ino {\textbar} mode {\textbar} lnk {\textbar} uid
{\textbar} gid) := @ ce}

{\sffamily\bfseries
Note}

{\sffamily
\texttt{stat()} returns file information. It is presented in the next chapter.}

\noindent
Co-expressions can be expensive. This is probably not a
good way to assign a series of values to a group of variables but it
demonstrates an interesting technique.

\section{Coroutines}

In conventional invocation, procedures have an asymmetric
relationship; when control is transferred from the caller to the
callee, the callee procedure starts execution at the top. Coroutines
have an equal relationship: when control is transferred from one
coroutine to another, execution starts from the point that
execution was suspended. This process is called resumption. The
\index{producer/consumer}producer/consumer problem is a good example of
procedures that have an equal relationship. Figure 4-1 shows how the
control flow between coroutines is different from that of conventional
procedures.

{\centering \par}

\begin{center}
\includegraphics[width=3.9902in,height=3.1701in]{ub-img/ub-img8.png}
\end{center}

{\sffamily\bfseries Figure 4-1:}
{\sffamily The Difference Between Procedures and Coroutines}

\bigskip

\noindent
Can you tell what the next example computes from its integer
command-line argument?

\bigskip

{\sffamily\bfseries
Listing 4-1}

{\sffamily\bfseries
Producer and Consumer Coroutines}

\iconcode{
procedure main(args) \\
\>   C1 := create consumer(args[1]) \\
\>   C2 := create producer(C1) \\
\>   @C2 \\
end \\
\ \\
procedure producer(ce) \\
\>   x := 1 \\
\>   repeat \{ \\
\>   \ \ \ val := x \^{} 2 \\
\>   \ \ \ ret := val @ ce {\textbar} break \\
\>   \ \ \ x +:= 1 \\
\>   \ \ \ \} \\
\>   @ \&main \\
end \\
\ \\
procedure consumer(limit) \\
\>   value := @ \&source \\
\>   repeat \{ \\
\>   \ \ \ \# process value \\
\>   \ \ \ if value {\textgreater} limit then break \\
\>   \ \ \ if value \% 2 = 0 then write(value) \\
\>   \ \ \ value := retval @ \&source \\
\>   \ \ \ \} \\
end
}

When producer resumes consumer, it passes \texttt{value}; the consumer
passes a return code (\texttt{retval}) back. \texttt{\&source} is the
coexpression that activated the current co-expression.

{\sffamily\bfseries
Note}

{\sffamily
This example doesn't mean the producer/consumer problem
should always be done with coroutines!}

\section{Permutations}

\index{permutations}We have seen one usage of
\index{map()}\texttt{map()}, where it transformed mixed-case strings to
all lowercase. In that type of usage, the first string \ is the one
that we are manipulating, and the other two arguments tell it how the
string is to be modified. Interesting results can be achieved by
treating the \textit{third} argument as the string to manipulate.
Consider this code:

\iconcode{
s := "abcde" \\
write(map("01234",
"43201", s))
}

What does this code example do? The transformation is:
\texttt{"4"} should be mapped to
\texttt{"a"},
\texttt{"3"} to
\texttt{"b"},
\texttt{"2"} to
\texttt{"c"},
\texttt{"0"} to
\texttt{"d"}, and
\texttt{"1"} to
\texttt{"e"}. When this mapping is applied
to \texttt{"01234"}, we get
\texttt{"decba"} - a permutation of the
string \texttt{s}! It is exactly the permutation that is suggested by
the first two arguments of \texttt{map()}. To arrange this sort of
permutation, all three strings must be the same size, and there must be
no repeated letters in the second string.


\begin{center}
\includegraphics[width=5.0in,height=1.8in]{ub-img/ub-img9.png}
\end{center}
{\sffamily\bfseries Figure 4-2:}
{\sffamily Permuting a string with the map() function}

\bigskip

Here is an example: In the USA, dates are represented with the month
coming first, as in 12/25/1998, but in many other places the day comes
first: 25/12/1998. This conversion is, of course, just a permutation;
we can do this with a simple map:

\iconcode{
map("Mm/Dd/XxYy",
"Dd/Mm/XxYy", date)
}

Here is another example. Unicon has a built-in random facility, the
\texttt{?} operator. Applied to a string or cset, it returns a random
character from the argument; applied to a structure, a random member of
that structure; and applied to an integer, a random integer between 1
and that number. This is a very useful feature and allows us to write
programs that shuffle cards or run simulations of things like rolling
dice.

By default in Unicon, the random sequence generated by \texttt{?} is
different for each run of the program. This is one of the few areas
where Unicon is deliberately different from Icon, which uses the same
seed each run by default. Icon's semantics is good
when debugging, because we want the program to behave predictably while
it is broken! However, in most applications that use random numbers,
such as games, different runs of the program should create different
numbers. The \index{random!number seed}random number seed is keyword
\texttt{\&random.} It can be assigned a value at the start of
\texttt{main()} in order to get Icon-style repeatability. Here's how
to assign it a number based on the current date and time.
Unicon's default semantics do something similar.

\iconcode{
\&random := map("sSmMhH",
"Hh:Mm:Ss", \&clock) + \\
\> \> \> \> map("YyXxMmDd",
"YyXx/Mm/Dd", \&date)
}

\noindent
The calls to \texttt{map()} remove punctuation characters from the
fixed-format strings produced by \texttt{\&clock} and \texttt{\&date}.
The resulting strings of digits are converted to integers, added, and
stored as a seed in \texttt{\&random}. Now every time the program is
run, the random number facility will be initialized with a different
number.

\section{Simulation}

A Galton Box demonstrates how balls falling through a
lattice of pegs will end up distributed binomially. A simulation of a
Galton box combines several of the techniques described previously.
Figure 4-3 is an illustration of the program's
screen.
\begin{center}
\includegraphics[width=1.75in,height=3.8201in]{ub-img/ub-img10.png}
\end{center}

{\sffamily\bfseries Figure 4-3:}
{\sffamily A Galton Box Simulation}

\bigskip

The simulation's output window is a good example of
Unicon's high-level graphics facilities. Graphics is a
broad area, discussed in Chapter 7 of this book; the on-line references
or the Icon graphics book (Griswold, Jeffery, Townsend 1998) contain
substantial additional details. Graphics are part of the system
interface. Some of the graphics functions used in this example include:

\begin{itemize}\itemsep0pt
\item \texttt{FillArc(x,y,width,height)} fills an ellipse defined by a
bounding rectangle. The shape is filled by the current foreground color
and/or fill pattern. The height defaults to be the same as the width,
producing a circle. Given additional arguments, \texttt{FillArc()
}fills parts of an ellipse similar to pieces of a pie in shape.
\item \texttt{WAttrib("attr")} or
\texttt{WAttrib("attr=value")}, the generic
getter/setter for window attributes.
In this case the attributes \texttt{fg} (foreground color) and
\texttt{drawop} (raster drawing operation) are set to various colors
and reversible output.
\item \texttt{Window("attr=value", ...)}
opens a window with characteristics specified by string
attribute values. The \texttt{WDelay(t)} function waits until
\texttt{t} milliseconds have passed. The \texttt{WDone()} function
waits for the user to dismiss the output window by pressing
\texttt{"q"} and then terminates the
program and closes the window.
\end{itemize}
Listing 4-2 contains the code for a simplified version of the
simulation. A couple elements of the image above are omitted in order
to make the example easy to follow. (Both this program and the one that
created the screen shot above are included on the
book's web site, http://unicon.org/book/)

{\sffamily\bfseries Listing 4-2}
{\sffamily\bfseries A Simple Galton Box Simulation}

\iconcode{
link graphics \\
global pegsize, height, width, pegsize2 \\
\ \\
procedure main(args) \\
local n, steps := 10 \\
\>   pegsize := 10 \\
\>   pegsize2 := pegsize * 2 \\
\>   n := integer(args[1]) {\textbar} 100 \\
\>   setup\_window(steps) \\
\>   every 1 to n do galton(steps) \\
\>   WDone() \\
end \\
\ \\
procedure setup\_window(n) \\
\ \\
local max, xpos, ypos, i, j \\
\>   \# Draw the n levels of pegs \\
\>   \# Pegboard size is 2n-1 square \\
\>   \# Expected max value of histogram is (n, n/2)/2\^{}n  \\
\>   \# ... approximate with something simpler? \\
\>   max := n*n/pegsize \\
\>   width := (2*n+1)*pegsize \\
\>   height := width + n*n/2*pegsize \\
\>   Window("size=" {\textbar}{\textbar}
width {\textbar}{\textbar} ","
{\textbar}{\textbar} height, "fg=grayish-white") \\
\>   WAttrib("fg=dark-grey") \\
\>   every i := 1 to n do \{ \\
\>   \ \ \ ypos := i * pegsize2 \\
\>   \ \ \ xpos := width/2 - (i - 1) * pegsize - pegsize/2 \\
\>   \ \ \ every j := 1 to i do \{ \\
\>   \ \ \ \ \ \ FillArc(xpos, ypos, pegsize, pegsize) \\
\>   \ \ \ \ \ \ xpos +:= pegsize2 \\
\>   \ \ \ \ \ \ \} \\
\>   \ \ \ \} \\
\>   \# Set up drawing mode to draw the falling balls \\
\>   WAttrib("fg=black") \\
\>   WAttrib("drawop=reverse") \\
end
\ \\
\# Do it! \\
procedure galton(n) \\
local xpos, ypos, oldx, oldy \\
\>   xpos := oldx := width/2 - pegsize/2 \\
\>   ypos := oldy := pegsize \\
\>   \# For every ball... \\
\>   every 1 to n do \{ \\
\>   \ \ \ if ?2 = 1 then xpos -:= pegsize \\
\>   \ \ \ else xpos +:= pegsize \\
\>   \ \ \ ypos +:= pegsize2 \\
\>   \ \ \ animate(oldx, oldy, xpos, ypos) \\
\>   \ \ \ oldx := xpos; oldy := ypos \\
\>   \ \ \ \} \\
\>   \# Now the ball falls to the floor \\
\>   animate(xpos, ypos, xpos, ypos + 40) \\
\>   animate(xpos, ypos+40, xpos, ypos + 200) \\
\>   \# Record this ball \\
\>   draw\_ball(xpos) \\
end
\ \\
procedure animate(xfrom, yfrom, xto, yto) \\
\>   animate\_actual(xfrom, yfrom, xto, yfrom, 4) \\
\>   animate\_actual(xto, yfrom, xto, yto, 10) \\
end
\ \\
\# Drawing op is already set to "reverse",
and fg colour is black. \\
procedure animate\_actual(xfrom, yfrom, xto, yto, steps) \\
local x := xfrom, y := yfrom, xstep, ystep, i, lastx, lasty \\
\>   xstep := (xto - xfrom)/steps \\
\>   ystep := (yto - yfrom)/steps \\
\>   every i := 1 to steps do \{ \\
\>   \ \ \ lastx := x; lasty := y \\
\>   \ \ \ FillArc(x, y, pegsize, pegsize) \\
\>   \ \ \ WDelay(1) \\
\>   \ \ \ FillArc(x, y, pegsize, pegsize) \\
\>   \ \ \ x +:= xstep; y +:= ystep \\
\>   \ \ \ \} \\
end
\ \\
procedure draw\_ball(x) \\
static ballcounts \\
initial ballcounts := table(0) \\
\>   ballcounts[x] +:= 1 \\
\>   FillArc(x, height-ballcounts[x]*pegsize, pegsize, pegsize) \\
end
}

\section{Arrays}
\index{array}
Unicon uses two different representations for a list: one of them, an
{\em array}, is optimized for fast access to the list-elements or for
interfacing with another language that expects a contiguous vector of
values. The size of an array is fixed when it is first created and
consists entirely of integer elements (or entirely of reals). If any
of the array elements is assigned a value of a different type, or the
size of the array changes (e.g. by using \texttt{delete} or
\texttt{push} etc.), the array is automatically converted into the
other form --- the more general representation of a list.

The general list representation occupies approximately twice as much
storage as an array of the same size and type on most implementations
of Unicon.

To create an array, call \texttt{list(n, x)}, where \texttt{x} is an
integer or real value and \texttt{n > 0}. If you want to create a more
general list, rather than an array (perhaps you know the list will
change its size in the future and you wish to avoid the cost of
converting the array when it does) then use list comprehension:
i.e. instead of
\iconcode{
  l := list(20000, 0)
}
\noindent use
\iconcode{
  l := [: (| 0) {\textbackslash} 20000 :]
}
\noindent list comprehension never produces an array.

Although the conversion from the array form to the more general
representation is automatic, when it is required, the conversion in the
other direction is not. To convert a general list into an array it may
be supplied as the only parameter to the \texttt{list} function. If
the list meets the requirements (either all the elements are integers
or all of them are reals) the function will return a new array that
consists of the values in the list. If the list does not meet the
requirements, the function returns the list parameter unchanged.

\section*{Summary}

Unicon is particularly powerful when language features are
combined. The ability to combine features in interesting ways is the
result of its novel expression semantics. Co-expressions add
substantial value to the concept of \index{generator}generators,
although most programs use them only sparingly. They fit well into a
philosophy that says that simple things should be easy to do \ldots and
complex things should be easy to do as well.

