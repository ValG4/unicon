\chapter{Installation}

The Downloads page of the Unicon web site (http://unicon.org) has
links to the binary distributions of Unicon and to the source code.
Unicon may be installed from a binary distribution for Intel based
Windows and MacOS platforms. Users on other platforms will usually have
to download the source code and build it themselves.  This will
generally require a supported C99 compiler and environment, such as a
make program compatible with GNU make.

Unicon's source code can be downloaded as a compressed
archive file with the extension .zip, or from a revision control
system. The revision control system sources are much more up to date.
There is a copy of the source code at
(https://sourceforge.net/projects/unicon) but Unicon development takes
place on GitHub (https://github.com/uniconproject/unicon): commits to the
main branch are reflected to SourceForge after a small delay.

Unicon is customized using the configure script in the top level directory
-- the options for customization are displayed by the {\tt configure --help}
command -- followed by a make command to build the software.  The top level
README file has more detailed instructions for the most popular platforms.
