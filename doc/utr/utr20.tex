\documentclass[letterpaper,12pt]{article}
\usepackage{utr}
\usepackage[margin=1in]{geometry}

\title{The Unicon JSON Library}
\author{Gigi Young and Clinton Jeffery}
\trnumber{20}
\date{October 16, 2020}

\begin{document}
\abstract{
This report describes a library for supporting conversion to and from 
JSON (JavaScript Object Notation) and their equivalent Unicon structures.
}
\maketitle

\section{Introduction}

JSON is a widely used lightweight data-interchange format [Crawford]
that supports nested, sequential and associative data structures.
This report describes two Unicon library
functions that allow for convenient data conversion
between JSON and Unicon.

\section{An Overview of the JSON Grammar}

The JSON grammar is straightforward. A JSON data structure is composed
of one or more JSON values, which can be either a string, number,
boolean true, boolean false, null, array, or object. An array contains
a list of JSON values, while an object is a set of string-value
pairs. In other words, an array is a list of values and an object is a
table (dictionary) that requires keys to be strings.
Arrays and objects may be arbitrarily nested within each other.

The manner in which the tokens of the JSON grammar are defined allows
a token's category to be determined from its first character. The JSON
tokens are string, number, true, false, null, and the operators used
for array, object, and their productions (\texttt{\{\}[]:,}). The
following table gives the possible first characters of each
non-operator JSON token.

\begin{center}
\begin{tabular}{ c | c }
 possible first characters  & JSON token \\
 \hline
 \{                         & object \\
 \verb [                    & array \\
 -0123456789                & number \\
 $\prime\prime$             & string \\
 t                          & true \\
 f                          & false \\
 n                          & null \\

\end{tabular}
\end{center}

\section{The Unicon JSON API}

The JSON library contains four thread-safe functions: \texttt{jtou()},
\texttt{jtous()}, \texttt{jtouf()} for converting from JSON to Unicon and 
\texttt{utoj()} for converting from Unicon to JSON.
The data equivalencies are as follows: 

\begin{center}
{\large\bf JSON to Unicon data type equivalency}
\end{center}

\begin{center}
\begin{tabular}{ c | c }
 JSON data type	& Unicon data type \\
 \hline
 object		& table \\
 array		& list \\
 string		& string \\
 number		& int, real \\
 true		& string - \texttt{"\_\_true\_\_"} \\
 false		& string - \texttt{"\_\_false\_\_"} \\
 null		& null value - \texttt{\&null} \\

\end{tabular}
\end{center}

\vspace{1em}

\begin{center}
\center{\large\bf Unicon to JSON data equivalency}
\end{center}

\begin{center}
\begin{tabular}{ c | c }
 Unicon data type			& JSON data type\\
 \hline
 table, set, cset, record, class	& object \\
 list					& array  \\
 string					& string \\
 int, real				& number \\
 string - \texttt{"\_\_true\_\_"}	& true   \\
 string - \texttt{"\_\_false\_\_"}	& false  \\
 null value - \texttt{\&null}		& null   \\

\end{tabular}
\end{center}

There are more Unicon data structures than JSON was designed to represent.
Consequently, string encodings
(\texttt{"\_\_unitable\_\_"}, \texttt{"\_\_uniset\_\_"}, 
\texttt{"\_\_unicset\_\_"},
\texttt{"\_\_unirecord\_\_"}, and \texttt{"\_\_uniclass\_\_"})
have been defined 
to differentiate the different types of Unicon structures 
represented by a JSON object (tables, sets, csets, records, and classes 
respectively). The JSON object Unicon-encodings are listed in the table 
below:

\begin{center}
\center{\large\bf JSON object Unicon-encodings}
\end{center}

\begin{center}
\begin{tabular}{ c | c }
 Unicon	data type	& JSON specification\\
 \hline
 table	& \texttt{\{"\_\_unitable\_\_":1, ...\}} \\
 set	& \texttt{\{"\_\_uniset\_\_":[...]\}} \\
 cset	& \texttt{\{"\_\_unicset\_\_":"..."\}} \\
 record	& \texttt{\{"\_\_unirecord\_\_":"RecordName","Field1":"Value1", ...\}}\\
 class	& \texttt{\{"\_\_uniclass\_\_":"ClassName","Field1":"Value1", ...\}} \\

\end{tabular}
\end{center}

\noindent
These encodings are strict and no type coercions will be performed. 
For example,

\begin{verbatim}
{"__unitable__":"1"}
\end{verbatim}

\noindent
would produce a Unicon table with key-value pair

\begin{verbatim}
T["__unitable__"] := "1"
\end{verbatim}

\noindent
instead of an empty table. 
Any JSON object that does not adhere to these Unicon-encodings will default
to a Unicon table. Furthermore, records and classes that adhere to 
Unicon-encoding syntax but whose constructors do not exist in the calling 
program will also be converted to Unicon tables. 

The Unicon table is a superset of what the JSON object can represent. 
A JSON object is a dictionary which only supports string keys, while
a Unicon table supports any valid Unicon type as a key. To support
JSON conversion of Unicon tables, the grammar of a JSON object is optionally
extended to accept any JSON value as a key. The extended JSON object becomes 
a set of value-value pairs instead of string-value pairs.

Although the aim of this library is to support inter-language data exchange, 
it is just as important to support all Unicon data types. The choosen compromise
allows Unicon users the decision of whether or not to adhere to strict JSON
with the parameter \texttt{strict}. \texttt{strict} is a parameter for 
every function in the JSON library API and defaults to \texttt{\&null}. If
\texttt{strict} is non-null, then library functions adhere to the JSON
specification. Otherwise, an extended-JSON specification which allows for
non-string key values for JSON objects is used.

\bigskip
\hrule\vspace{0.1cm}
\noindent
{\tt\bf jtou(s, strict:\&null, mode:\&null, error:\&errout)} \hfill {\tt\bf JSON to Unicon}

\vspace{0.1cm}
\noindent
\texttt{jtou(s)} generates the equivalent Unicon data structure(s) to
parameter \texttt{s}, a JSON-encoded string or JSON filename. 
Parameter \texttt{strict} enforces strict adherence to the JSON specification 
if non-null.
Parameter \texttt{mode} can be specified as \texttt{"s"} (see \texttt{jtous()}) 
to force JSON-encoded string conversion or as \texttt{"f"} 
(see \texttt{jtouf()}) to force JSON file conversion.
\texttt{jtou(s)} tries to first process \texttt{s} as a string, then as a 
filename if \texttt{mode} is not specified. 
Parameter \texttt{error} allows the user to specify a file or filestream other 
than standard error for error output. 
This function fails if \texttt{s} is not valid JSON, depending on parameter 
\texttt{strict}.
It should be noted that error messages may be misleading if \texttt{mode}
is not specified.

To conform with other JSON libraries, trailing commas are allows for JSON arrays
and objects.

\bigskip
\hrule\vspace{0.1cm}
\noindent
{\tt\bf jtous(s, strict:\&null, error:\&errout)} \hfill {\tt\bf JSON string to Unicon}

\vspace{0.1cm}
\noindent
\texttt{jtous(s)} (equivalent to \texttt{jtou(s,"s")}) generates the
equivalent Unicon data structure(s) to parameter \texttt{s}, a JSON-encoded 
string. 
Parameter \texttt{strict} enforces strict adherence to the JSON specification 
if non-null.
Parameter \texttt{error} allows the user to specify a file or filestream other 
than standard error for error output. 
This function fails if \texttt{s} is not valid JSON, depending on parameter 
\texttt{strict}.

To conform with other JSON libraries, trailing commas are allows for JSON arrays
and objects.

\bigskip
\hrule\vspace{0.1cm}
\noindent
{\tt\bf jtouf(s, strict:\&null, error:\&errout)} \hfill {\tt\bf JSON file to Unicon}

\vspace{0.1cm}
\noindent
\texttt{jtouf(s)} (equivalent to \texttt{jtou(s,"f")}) generates the 
equivalent Unicon data structure(s) to parameter \texttt{s}, a JSON filename. 
Parameter \texttt{strict} enforces strict adherence to the JSON specification 
if non-null.
Parameter \texttt{error} allows the user to specify a file or filestream other 
than standard error for error output. 
This function fails if \texttt{s} is not valid JSON, depending on parameter 
\texttt{strict}.

To conform with other JSON libraries, trailing commas are allows for JSON arrays
and objects.

\bigskip
\hrule\vspace{0.1cm}
\noindent
{\tt\bf utoj(u, strict:\&null, error:\&errout)} \hfill {\tt\bf Unicon to JSON}

\vspace{0.1cm}
\noindent
\texttt{utoj(u)} takes a Unicon value, \texttt{u}, and returns a corresponding 
JSON-formatted string. 
Parameter \texttt{strict} enforces strict adherence to the JSON specification 
if non-null.
Parameter \texttt{error} allows the user to specify a filestream other than 
standard error for error output. 
%This function should not fail unless there's a bug...

\pagebreak
\section{Examples}

Unicon's JSON library is straightforward. \texttt{jtou()}, \texttt{jtous()},
or \texttt{jtouf()} are used to convert JSON to Unicon data types and 
\texttt{utoj()} is used to convert Unicon data types to JSON. 
\texttt{jtou(s,"s")} is equivalent to \texttt{jtous(s)} and \texttt{jtou(s,"f")}
is equivalent to \texttt{jtouf(s)}. \texttt{jtou(s)} will convert \texttt{s}
whether it is a JSON-encoded string or JSON filename as long as the JSON is
valid. However, one may want to use the specific functions \texttt{jtous()}
and \texttt{jtouf()} in the case of error handling accuracy. 

\bigskip
\hrule\vspace{0.1cm}
\bigskip
\noindent
The following program converts a JSON data structure which
consists of an object and an array to a Unicon list containing a table and
list. The \texttt{jtou*()} family of functions are generators, which allows
the use of the list constructor.

\begin{verbatim}
link json
procedure main()
   u := [: jtou("{\"one\":1, \"two\":2, \"three\":3}_
       [true, false, null, 1.23e3]") :]
   x := u[1]
   y := u[2]
   write(x["one"])
   write(x["two"])
   write(x["three"])
   every i := 1 to *y do write(y[i])
end
\end{verbatim}

\vspace{0.1cm}
\noindent
The output of which is:

\begin{verbatim}
1
2
3
__true__
__false__
 1230.0
\end{verbatim}

\bigskip
\hrule\vspace{0.1cm}
\bigskip
\noindent
%\vspace{-1em}
%\noindent
This simple program converts a Unicon table to a JSON object

\begin{verbatim}
link json
procedure main()
   T := table()
   T["one"] := 1
   T["two"] := 2
   T["list"] := [1,2,3]
   T["table"] := table()
   write(utoj(T))
end
\end{verbatim}

\vspace{0.1cm}
\noindent
and outputs:

\begin{verbatim}
{"__unitable__":1,"two":2,"table":{"__unitable__":1},"one":1,"list":[1,2,3]}
\end{verbatim}

The extraction of keys from Unicon tables occurs in a random manner,
resulting in JSON data that does not follow the order in which the Unicon
table was constructed. However, lists are sequential in Unicon, so the
resulting ordering of the JSON array is retained. 

\bigskip
\hrule\vspace{0.1cm}
\bigskip
\noindent
The following program performs a class conversion 

\begin{verbatim}
link json

class Person(name, age, gender)
   #
   # some methods here
   #
end

procedure main()
   person := Person("Joe", 42, "Male")
   write(utoj(person))
end
\end{verbatim}

\vspace{0.1cm}
\noindent
which outputs:

\begin{verbatim}
{"__uniclass__":"Person","name":"Joe","age":42,"gender":"Male"}
\end{verbatim}

\bigskip
\hrule\vspace{0.1cm}
\bigskip
\noindent
The following program performs a Unicon-to-JSON-back-to-Unicon conversion and
is more complicated than the previous examples. This shows that the JSON
library can be used to store Unicon data. Unicon's JSON library does not
store object references, but can recreate structually equivalent objects.

\begin{verbatim}
link json
link ximage

record R(a,b,c)

class S(d,e,f)
end

procedure main()
   c := S()
   c.d := "a"
   c.e := "b"
   c.f := "c"
   r := R(1,2,3)
   s := set(["abc"])
   s2 := set(["__unicset__","__uniset__","__uniclass__","__unirecord__"])
   t := table()
   t['a'] := 'bc'
   t["\177b"] := 2
   t["c"] := table()
   t["c"]["\^cd"] := 3.89e-4
   t[1] := r
   t[2] := c
   t[s] := set(["I'm a set element"])
   t["__uniset__"] := s2
   l := [t, s]

   write("Before encoding:",ximage(l))
   X := utoj(l)
   write("\nEncoded JSON: ",X)
   y := jtous(X)
   write("\nAfter encoding:",ximage(y),"\n")
end

\end{verbatim}

\vspace{0.1cm}
\noindent
This program produces the output:

\begin{verbatim}
Before encoding:L4 := list(2)
   L4[1] := T1 := table(&null)
      T1[1] := R_R_1 := R()
         R_R_1.a := 1
         R_R_1.b := 2
         R_R_1.c := 3
      T1[2] := S_1 := S()
         S_1.d := "a"
         S_1.e := "b"
         S_1.f := "c"
      T1["__uniset__"] := S2 := set()
         insert(S2,"__uniclass__")
         insert(S2,"__unicset__")
         insert(S2,"__unirecord__")
         insert(S2,"__uniset__")
      T1["c"] := T2 := table(&null)
         T2["\x03d"] := 0.000389
      T1["\db"] := 2
      T1['a'] := 'bc'
      T1[{S1 := set()
            insert(S1,"abc")
         S1}] := S3 := set()
         insert(S3,"I'm a set element")
   L4[2] := S1

Encoded JSON: [{"__unitable__":1,"\u007Fb":2,2:{"__uniclass__":"S","d":"a","
e":"b","f":"c"},"__uniset__":{"__uniset__":["__uniclass__","__uniset__","__u
nirecord__","__unicset__"]},1:{"__unirecord__":"R","a":1,"b":2,"c":3},{"__un
iset__":["abc"]}:{"__uniset__":["I'm a set element"]},{"__unicset__":"a"}:{"
__unicset__":"bc"},"c":{"__unitable__":1,"\u0003d":0.000389}},{"__uniset__":
["abc"]}]

After encoding:L24 := list(2)
   L24[1] := T7 := table(&null)
      T7[1] := R_R_3 := R()
         R_R_3.a := 1
         R_R_3.b := 2
         R_R_3.c := 3
      T7[2] := S_2 := S()
         S_2.d := "a"
         S_2.e := "b"
         S_2.f := "c"
      T7["__uniset__"] := S4 := set()
         insert(S4,"__uniclass__")
         insert(S4,"__unicset__")
         insert(S4,"__unirecord__")
         insert(S4,"__uniset__")
      T7["c"] := T13 := table(&null)
         T13["\x03d"] := 0.000389
      T7["\db"] := 2
      T7['a'] := 'bc'
      T7[{S5 := set()
            insert(S5,"abc")
         S5}] := S6 := set()
         insert(S6,"I'm a set element")
   L24[2] := S7 := set()
      insert(S7,"abc")
\end{verbatim}

\newpage
\section*{References}

[Crawford] anonymous, but owing to Douglas Crawford.
"Introducing JSON", json.org.

\end{document}
